\chapter{Įvadas}

\section{Psichologijos vieta mokslų sistemoje}

\label{tema:psichologijos_mokslas}

Žodis „psichologija“ sudarytas iš dviejų graikų kalbos žodžių: 
\emph{psyche} (gyvybė, siela) ir \emph{logos} (mokslas). Taigi, verčiant
pažodžiui, psichologija yra mokslas apie sielą.

Galime išskirti buitinę ir mokslinę psichologiją. Mokslinei psichologijai
būdinga:
\begin{itemize}
  \item metodiškumas – žinios kaupiamos sąmoningai, tikslingai atliekant
    tyrimus ir bandymus;
  \item sistemingumas – bandoma sukurti sistemą paaiškinančią visus 
    stebimus reiškinius (žinios yra apibendrinamos, siekiama rasti 
    universalius principus, priežasties – pasekmės ryšius ir t.t.)
  \item tikslumas – visos naudojamos sąvokos turi konkrečią, tiksliai
    apibrėžtą prasmę.
\end{itemize}
Buitinei psichologijai būdinga:
\begin{itemize}
  \item padrikumas – žinios kaupiamos iš atsitiktinių pastebėjimų, 
    stebint save ir kitus;
  \item žinios konkrečios, pritaikytos konkrečioms atskiroms situacijoms;
  \item vartojamos sąvokos turi „išplaukusią“, be konkrečių ribų 
    semantinę reikšmę.
\end{itemize}

\emph{Papildymas iš vadovėlio:}

Psichologija priklauso mokslų apie žmogų sistemai, todėl ji susijusi su
visais mokslais, kurie vienu ar kitu aspektu liečia žmogaus egzistencijos 
problemas: \emph{filosofija}, \emph{sociologija}, \emph{biologija},
\emph{pedagogika} ir kt. Specifinė psichologijos ryšių su kitais mokslais
priežastis yra dar ta (be tyrimo objektų gretimumo), kad kad visus mokslus
kuria žmonės savo psichinių ir fizinių pastangų dėka - jusdami ir suvokdami
tikrovę, mąstydami ir apibendrindami, tam tikrų motyvų skatinami 
pasirinkdami tyrimų kryptis. Vadinasi, bet koks mokslinis pažinimas yra 
sudėtinga psichinė veikla, kuri gali būti psichologijos mokslo objektas.
Dėl to psichologijos žinios yra kiekvieno mokslo objektas.

\section{Šiuolaikinės psichologijos šakos}

\label{tema:psichologijos_sakos}

Šiuolaikinė psichologija, aprėpdama vis įvairesnes žmonių psichinio gyvenimo
sritis, kaip ir kiti mokslai, diferencijuojasi į atskiras disciplinas, 
besiskiriančias tyrimo objektais, metodais ir rezultatų pritaikymo sferomis.

Pagrindinės šakos:
\begin{itemize}
  \item \emph{bendroji psichologija} apima visų psichologijos disciplinų 
  sistemą. Ji tiria bendriausius normalaus suaugusio žmogaus psichikos 
  dėsningumus
  \item \emph{socialinė psichologija} yra mookslas apie žmonių grupėms 
  (pvz., moklsivių klasei, statybininkų brigadai, partijai ir t.t)
  būdingus reiškinius
  \item \emph{individo psichologija} tiria atskirų žmonių psichiką
  \item \emph{amžiaus tarpsnių psichologija}  tiria atskiro žmogaus 
  psichiką tam tikrą amžsiaus tarpsnį. Ji skirstoma į vaiko, paauglio,
  jaunuolio, subrendusio ir seno žmogaus psichologiją
  \item \emph{zoopsichologija} tyrinėja žemesniųjų gyvųnų psichikos 
  reiškinius (nuo pirmųjų psichikos pasireiškimų pirmuonių, vabzdžių 
  pasaulyje iki žmogbeždžionių psichikos)
  \item \emph{darbo ir inžinerinė psichologija} tyrinėja profesines asmenybės
  ypatybes, asmenybės tobulėjimo gamybinėje veikloje sąlygas ir kt.
  \item \emph{pedagoginė psichologija} tiria mokymo ir auklėjimo bei 
  mokymosi ir saviauklos psichologinius dėsningumus
  \item \emph{kūrybos p-ija}
  \item \emph{sporto p-ija}
  \item \emph{medicinos p-ija}
  \item \emph{teisės p-ija}
  \item \emph{prekybos p-ija}
\end{itemize}
Visas šiuolaikinės psichologijos šakas būtų sudėtinga išvardinti.

\section{Psichologijos tyrimų objektas}

\label{tema:psichikos_samprata}

FIXME: Psichika – Vidinis žmogaus pasaulis, jo samonė, protavimas. Dažnai 
psichika gretinama su siela.
TODO: Psichikos multidimensiškumas – apibrėžti.

\Gls{psichologija} nagrinėja tris pagrindines psichinių reiškinių grupes:

\begin{enumerate}
  \item psichiniai procesai (\ref{tema:psichiniai_procesai} skyrius):
    \begin{itemize}
      \item \glspl{pojutis},
      \item suvokimas (ang. perception),
      \item atmintis,
      \item mąstymas,
      \item vaizduotė;
    \end{itemize}
  \item psichinės būsenos (\ref{tema:psichines_busenos} skyrius):
    \begin{itemize}
      \item dėmesys,
      \item emocinė aktyvacija;
    \end{itemize}
  \item TODO: individualios asmenybės savybės:
    \begin{itemize}
      \item psichinių procesų ir būsenų individualios variacijos,
      \item „grynai“ individualios savybės:
        \begin{enumerate}
          \item temperamentas,
          \item gebėjimas,
          \item charakteris.
        \end{enumerate}
    \end{itemize}
\end{enumerate}

\section{Gyvybės psichikos evoliucija}

\label{tema:gyvybes_psichikos_evoliucija}

Gyvybės psichikos evoliucijoje svarbūs trys etapai: gyvosios gamtos 
atsiradimas iš negyvosios, perėjimas nuo biologinio į psichinį atspindėjimą,
perėjimas nuo gyvūno psichikos į žmogaus sąmonę.
Pagrininiai biologinės evoliucijos veiksniai buvo genetinis kintamumas
(mutacijos genetinėje medžiagoje) ir naturalioji atranka. Individuali 
atranka virto individualia-grupine. Grupinė atranka skatino gasinių signalų
suvokimo, įrankių gamybos ir kitų procesų tobulėjimą. Žmogus vystėsi iš 
primatų tipo protėvių natūraliomis (biologinėmis) sąlygomis, o vėliau 
prasidėjo ir kultūrinė evoliucija. Atsirado žmogus, gebantis savo 
individualią patirtį jungti su visuomenine patirtimi. Prasidėjo žmogaus
 kurltūrinė evoliucija, ypač paskatinusi psichikos vystymąsi.

\section{Individuali zmogaus psichikos raida}

\label{tema:psichikos_raida}

Psichinės raidos periodai, kai iš esmės keičiasi individo santykiai
su fizine ir socialine aplinka, jo reakcijos, kai vyksta lūžiai asmenybėje,
vadinami \emph{kritiniais}. Kritiniai periodai būna pereinant iš vieno 
amžiaus tarpsnio į kitą, kai formuojasi kokybiškai nauji asmenybės 
bruožai, keičiasi savęs įvertinimas ir naujų elementų įgyja „aš“ sistema.
Juos gali sukelti ir atskiri įvykiai (pvz., artimo žmogaus netekimas, karas).


Svarbesni psichinės raidos periodai:
\begin{itemize}
  \item pirmasis ir ryškiausias žmogaus gyvenimo periodas yra pirmieji 
  mėnesiai po gimimo. Pradeda funkcionuoti dar vaisiuje susiformavę 
  refleksai, prasideda tiesioginė sąveika su daug variablesne aplinka
  \item antrasis periodas prasideda pirmųjų metų pabaigoje. Bejėgis, nuo
  suaugusiųjų priklausomes kūdikis mokosi pats atlikti sudėtingus judesius,
  vaikščioti. Prasiplečia pažinimas, vystosi vaizdinių bei žodžių užuomazgos
  \item trečiasis prasideda maždaug po dviejų metų, kurio pagrindinis 
  bruožas - kalbos išmokimas. Vaiko dėmesys jau nukreiptas ne tik į išorę,
  bet ir į save
  \item ketvirtuoju periodu (7-ais gyvenimo metais) išryškėja vaiko „vidinė
  pozicija“: vaikas pradeda suvokti save kaip socialinį individą.
  \item penktasis periodas - lytinio brendimo laikotarpis (12-17 metai).
  Intensyviai vystosi saviraiška ir savirealizacija. Paaugliai jau suvokia 
  save kaip asmenybę, kurią lygina su susikurtu idealu. Jie aktyviai
  ieško savo vietos visuomenėje ir suvokia gyvenimo prasmė, kuria ateities
  planus.
  \item sestasis periodas - brandos metas. Perėjimas į jį nėra ryškus, nors 
  jis susijęs su daug svarbių asmeninio gyvenimo įvykių (sukuriama šeima,
  pradedama dirbti, gimta vaikai ir t.t.)
  \item septintasis periodas yra senėjimo pradžia: silpnėja reprodukcinės 
  funkcijos, iš šeimos pasitraukia vaikai, atsiranda sunkumų profesinėje 
  veikloje, sustiprėja vienišumo tendencijos, žmogus darosi irzlesnis
  \item aštuntasis periodas - senatvė. Žmogus priverstas palikti profesinį
  darbą, suyra šeima (dėl sutuoktinio mirties), susilpnėja socialiniai ryšiai,
  dažniau galvojama apie gyvenimo pabaigą.
\end{itemize}

Čia pateiktos periodų charakteristikos, be abejo, turi daug išimčių bei yra 
nepilnos, nes skiriasi žmonių brendimo bei senėjimo tempai ir pobūdis.

\section{Psichologijos metodai ir jų praktinis taikymas}

\label{tema:psichologijos_metodai}

\subsection{Empirinių duomenų rinkimo ir matavimo būdai}

Empiriniai (patyrimu pagrįsti) faktai yra informacija apie individo elgseną
įvairiose situacijose. Stabint natūralią elgseną yra naudojamas 
\emph{išprovokuotas stebėjimas}, kai stebėtojas sudaro situacijas, panašias 
į natūralias. Šis būdas dažnai taikomas stebint socialinę elgseną. Dar 
vienas natūralios elgsenos stebėjimo būdų - \emph{stebėjimas dalyvaujant}, 
kai stebėtojas bando pasidaryti grupės nariu ir stebėti elgseną „iš grupės 
vidaus“.

Dažnai apribojamas elgseną veikiančių faktorių kiekis, praktikuojami stebėjimai
specialiomis arba kontroliuojamomis sąlygomis. Tam taikomos įvairios užduotys,
kurios turėtų nukreipti tiriamojo veiklą ir sudaryti sąlygas stebėti tik tam
tikrą su užduotimi susijusią elgsenos rūšį.

Taip pat taikoma ir \emph{tiriamųjų apklausa}, kai stebėtojas pateikia klausimus
tiriamąjam. Klausimai būna pateikiami įvairiomis formomis: atvirais ar 
uždarais anketų klausimais, pokalbio ar interviu metu ir kt.

\subsection{Stebėjimo (aprašomoji) strategija}

\emph{Stebėjimo strategija} - tai toks mokslinio tyrimo organizavimas, kai 
nevarijuojant sąlygų siekiama gauti objektyvių faktų apie žmogaus ar gyvūno 
elgseną. Atliekant tokio pobūdžio tyrimą, gali būti taikomi visi empirinių 
duomenų rinkimo būdai. Dažnai duomenys tokiems tyrimams gaunami stebint 
natūralią elgseną. Kadangi, taikant šią strategiją, empirianiai duomenys 
dažnai renkami nekontroliuojant veiksnių ir sąlygų, galinčių turėti įtakos 
registruojamai elgsenai, tai tyrėjas negali paaiškinti, kas lėmė jo 
užregistruotų faktų ypatumus. Todėl pagal daromų išvadų pobūdį ši strategija 
vadinama aprašomąja. Yra laikomasi nuomonės, kad stebėjimai yra pirmoji tiek
mokslo apskritai, tiek konkretaus tyrimo fazė, parengianti pamatą 
ekspermentiniams tyrimams.

\subsection{Koreliaciniai tyrimai}

Koreliaciniais tyrimais siekiama įvertinti ryšį tarp dviejų ar daugiau 
veiksnių (kintamųjų). Hipotezė apie reiškinių ryšį gali būti patikrinta ir 
įvertinta koreliaciniu tyrimu. 

Ši strategija reikalauja, kad: 
\begin{enumerate}
  \item tie reiškiniai, arba veiksniai, būtų įvertinti kiekybiškai
  \item tie kiekybiniai įvertinimai būtų gauti tiriant tuos pačius individus
\end{enumerate}

Šis tyrimas parodo tik ryšio tarp reiškinių buvimą, tačiau neparodo, kuris
reiškinys kurį įtakoja ar veikia. 

\subsection{Ekspermentinė strategija}

Pagrindinis šios mokslinių tyrimų strategijos ypatumas yra tas, kad tyrėjas 
sukelia norimą tirti reiškinį ir, tikslingai keisdamas tam tikras sąlygas, 
stebi, kaip priklausomai nuo jo vieksmų kinta kiti su tuo susiję reiškiniai.
Dauguma ekspermentų dabar atliekama specialiose psichologijos laboratorijose,
kur yra geriausios sąlygos kontroliuoti kintamuosius. Tokie ekspermentai 
vadinami \emph{laboratoriniais}. Tačiau gali būti ekspermentuojama ir 
natūralioje individo aplinkoje. Tokie ekspermentai vadinami 
\emph{natūraliaisiais}.