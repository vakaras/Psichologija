\chapter{Įvadas}

\section{Psichologijos vieta mokslų sistemoje}

\label{tema:psichologijos_mokslas}

Žodis „\gls{psichologija}“ sudarytas iš dviejų graikų kalbos žodžių: 
\emph{psyche} (gyvybė, siela) ir \emph{logos} (mokslas). Taigi, verčiant
pažodžiui, \gls{psichologija} yra mokslas apie sielą.

Galime išskirti buitinę ir mokslinę psichologiją. Mokslinei psichologijai
būdinga:
\begin{itemize}
  \item metodiškumas – žinios kaupiamos sąmoningai, tikslingai atliekant
    tyrimus ir bandymus;
  \item sistemingumas – bandoma sukurti sistemą paaiškinančią visus 
    stebimus reiškinius (žinios yra apibendrinamos, siekiama rasti 
    universalius principus, priežasties – pasekmės ryšius ir t.t.)
  \item tikslumas – visos naudojamos sąvokos turi konkrečią, tiksliai
    apibrėžtą prasmę.
\end{itemize}
Buitinei psichologijai būdinga:
\begin{itemize}
  \item padrikumas – žinios kaupiamos iš atsitiktinių pastebėjimų, 
    stebint save ir kitus;
  \item žinios konkrečios, pritaikytos konkrečioms atskiroms situacijoms;
  \item vartojamos sąvokos turi „išplaukusią“, be konkrečių ribų 
    semantinę reikšmę.
\end{itemize}

\section{Šiuolaikinės psichologijos šakos}

\label{tema:psichologijos_sakos}

Psichologija gali būti mokslinė ir taikomoji. Mokslinės psichologijos
šakos:
\begin{itemize}
  \item bendroji – tiria sveiko, suaugusio žmogaus psichikos reiškinius;
  \item raidos – nagrinėja žmogaus elgesio, psichikos vystymąsi per jo
    gyvenimą;
  \item socialinė – tiria žmogaus elgesį grupėje;
  \item asmenybės – tiria asmenybės tapsmo procesą (kaip žmogus tampa
    asmenybe, kaip susiformuoja jo vertybės, norai, pastovus elgesys,
    tam tikras mąstymo būdas);
  \item anomalios raidos psichologija – tiria įvairius nukrypimus nuo to,
    ką mes laikome norma.
\end{itemize}
Taikomosios psichologijos šakos:
\begin{itemize}
  \item klinikinė – patologinių reiškinių nustatymas, jų gydymas bei
    pagalba sveikiems žmonėms, kurie kasdieniame gyvenime turi tam tikrų
    sunkumų (Naudojama psichoterapija ir konsultavimas. Šie psichologai
    dirba psichiatrijos ir paprastose ligoninėse, psichoterapijos 
    centruose, konsultacinėse tarnybose);
  \item pedagoginė – švietimo;
  \item organizacinė – psichologijos žinių taikymas versle, pramonėje
    (padeda suprasti žmogaus santykius su viršininkais, bendradarbiais,
    nuo ko priklauso žmogaus pasitenkinimas darbu, kaip galima sumažinti
    įtampą darbe, kaip pasiekti, kad žmonės geriau dirbtų).
\end{itemize}

\section{Psichologijos tyrimų objektas}

\label{tema:psichikos_samprata}

\Gls{psichika} – \glsentrydesc{psichika}. Psichika susideda iš tokių 
komponentų:
\begin{description}
  \item[sąmonė] – gamtinės ir visuomeninės aplinkos atspindėjimas, dėsnių
    pažinimas, būsimų tikrovės reiškinių prognozavimas;
  \item[savimonė] – savo asmenybės ir jos santykių su aplinka atspindėjimas;
  \item[nesąmoningų, arba neįsisąmonintų reiškinių sritis] – sapnai, 
    pamirštų praeities išgyvenimų vaizdiniai, ėjimo ir panašūs įgūdžiai;
    reiškiniai apie kurių vyksmą žmogus arba visai nežino, arba žino
    labai neapibrėžtai.
\end{description}

TODO: Psichikos multidimensiškumas – apibrėžti.

\Gls{psichologija} nagrinėja tris pagrindines psichinių reiškinių grupes:
\begin{enumerate}
  \item psichiniai procesai (\ref{tema:psichiniai_procesai} skyrius):
    \begin{itemize}
      \item \glspl{pojutis},
      \item suvokimas (ang. perception),
      \item atmintis,
      \item mąstymas,
      \item vaizduotė;
    \end{itemize}
  \item psichinės būsenos (\ref{tema:psichines_busenos} skyrius):
    \begin{itemize}
      \item dėmesys,
      \item emocinė aktyvacija;
    \end{itemize}
  \item TODO: individualios asmenybės savybės:
    \begin{itemize}
      \item psichinių procesų ir būsenų individualios variacijos,
      \item „grynai“ individualios savybės:
        \begin{enumerate}
          \item temperamentas,
          \item gebėjimas,
          \item charakteris.
        \end{enumerate}
    \end{itemize}
\end{enumerate}
