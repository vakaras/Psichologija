\chapter{Įvadas}

\section{Psichologijos vieta mokslų sistemoje}

\label{tema:psichologijos_mokslas}

Žodis „\gls{psichologija}“ sudarytas iš dviejų graikų kalbos žodžių: 
\emph{psyche} (gyvybė, \gls{siela}) ir \emph{logos} (mokslas). Taigi, 
verčiant pažodžiui, \gls{psichologija} yra mokslas apie sielą.

Galime išskirti buitinę ir mokslinę psichologiją. Mokslinei psichologijai
būdinga:
\begin{itemize}
  \item metodiškumas – žinios kaupiamos sąmoningai, tikslingai atliekant
    tyrimus ir bandymus;
  \item sistemingumas – bandoma sukurti sistemą, paaiškinančią visus 
    stebimus reiškinius (žinios yra apibendrinamos, siekiama rasti 
    universalius principus, priežasties – pasekmės ryšius ir t.~t.)
  \item tikslumas – visos naudojamos sąvokos turi konkrečią, tiksliai
    apibrėžtą prasmę.
\end{itemize}
Buitinei psichologijai būdinga:
\begin{itemize}
  \item padrikumas – žinios kaupiamos iš atsitiktinių pastebėjimų, 
    stebint save ir kitus;
  \item žinios konkrečios, pritaikytos konkrečioms atskiroms situacijoms;
  \item vartojamos sąvokos turi „išplaukusią“, be konkrečių ribų 
    semantinę reikšmę.
\end{itemize}

Psichologija priklauso mokslų apie žmogų sistemai, todėl ji susijusi su
visais mokslais, kurie vienu ar kitu aspektu liečia žmogaus egzistencijos 
problemas: \emph{filosofija}, \emph{sociologija}, \emph{biologija},
\emph{pedagogika} ir kt. Specifinė psichologijos ryšių su kitais mokslais
priežastis yra dar ta (be tyrimo objektų gretimumo), kad visus mokslus
kuria žmonės savo psichinių ir fizinių pastangų dėka – jusdami ir suvokdami
tikrovę, mąstydami ir apibendrindami, tam tikrų motyvų skatinami 
pasirinkdami tyrimų kryptis. Vadinasi, bet koks mokslinis pažinimas yra 
sudėtinga psichinė veikla, kuri gali būti psichologijos mokslo objektas.
Dėl to psichologijos žinios yra kiekvieno mokslo objektas.

\section{Šiuolaikinės psichologijos šakos}

\label{tema:psichologijos_sakos}

Psichologija gali būti mokslinė ir taikomoji. Mokslinės psichologijos
šakos:
\begin{itemize}
  \item \emph{bendroji} – apima visų psichologijos disciplinų sistemą,
    tiria bendriausius normalaus suaugusio žmogaus psichikos dėsningumus;
  \item \emph{raidos} – nagrinėja žmogaus elgesio, psichikos vystymąsi per 
    jo gyvenimą;
  \item \emph{amžiaus tarpsnių} – tiria atskiro žmogaus psichiką tam tikrą
    amžiaus tarpsnį; skirstoma į vaiko, paauglio, jaunuolio, subrendusio
    ir seno žmogaus psichologiją;
  \item \emph{socialinė} – tiria žmogaus elgesį grupėje;
  \item \emph{asmenybės} – tiria asmenybės tapsmo procesą (kaip žmogus tampa
    asmenybe, kaip susiformuoja jo vertybės, norai, pastovus elgesys,
    tam tikras mąstymo būdas);
  \item \emph{anomalios raidos psichologija} – tiria įvairius nukrypimus 
    nuo to, ką mes laikome norma;
  \item \emph{zoopsichologija} – tiria žemesniųjų gyvūnų psichikos 
    reiškinius (nuo pirmųjų psichikos pasireiškimų pirmuonių ir vabzdžių
    pasaulyje iki žmogbeždžionių psichikos).
\end{itemize}
Taikomosios psichologijos šakos:
\begin{itemize}
  \item \emph{klinikinė} – patologinių reiškinių nustatymas, jų gydymas bei
    pagalba sveikiems žmonėms, kurie kasdieniame gyvenime turi tam tikrų
    sunkumų (Naudojama psichoterapija ir konsultavimas. Šie psichologai
    dirba psichiatrijos ir paprastose ligoninėse, psichoterapijos 
    centruose, konsultacinėse tarnybose);
  \item \emph{pedagoginė} – tiria mokymo ir auklėjimo bei mokymosi ir saviauklos
    psichologinius dėsningumus;
  \item \emph{organizacinė} – psichologijos žinių taikymas versle, pramonėje
    (padeda suprasti žmogaus santykius su viršininkais, bendradarbiais,
    nuo ko priklauso žmogaus pasitenkinimas darbu, kaip galima sumažinti
    įtampą darbe, kaip pasiekti, kad žmonės geriau dirbtų).
\end{itemize}

\section{Psichologijos tyrimų objektas}

\label{tema:psichikos_samprata}

\Gls{psichika} – \glsentrydesc{psichika}. Psichika susideda iš tokių 
komponentų:
\begin{description}
  \item[sąmonė] – gamtinės ir visuomeninės aplinkos atspindėjimas, dėsnių
    pažinimas, būsimų tikrovės reiškinių prognozavimas;
  \item[savimonė] – savo asmenybės ir jos santykių su aplinka atspindėjimas;
  \item[nesąmoningų, arba neįsisąmonintų reiškinių sritis] – sapnai, 
    pamirštų praeities išgyvenimų vaizdiniai, ėjimo ir panašūs įgūdžiai;
    reiškiniai, apie kurių vyksmą žmogus arba visai nežino, arba žino
    labai neapibrėžtai.
\end{description}

TODO: Psichikos multidimensiškumas – apibrėžti.

\Gls{psichologija} nagrinėja tris pagrindines psichinių reiškinių grupes:
\begin{enumerate}
  \item psichiniai procesai (\ref{tema:psichiniai_procesai} skyrius):
    \begin{itemize}
      \item \glspl{pojutis},
      \item suvokimas (ang. perception),
      \item atmintis,
      \item mąstymas,
      \item vaizduotė;
    \end{itemize}
  \item psichinės būsenos (\ref{tema:psichines_busenos} skyrius):
    \begin{itemize}
      \item dėmesys,
      \item emocinė aktyvacija;
    \end{itemize}
  \item individualios asmenybės savybės:
    \begin{itemize}
      \item psichinių procesų ir būsenų individualios variacijos,
      \item „grynai“ individualios savybės
        (\ref{tema:individualios_savybes} skyrius):
        \begin{enumerate}
          \item temperamentas,
          \item gebėjimas,
          \item charakteris.
        \end{enumerate}
    \end{itemize}
\end{enumerate}

\section{Kūno ir sielos santykis}

\label{tema:kuno_sielos_santykis}

Žmogus savo kūnu priklauso materialiam pasauliui, o siela sugebėjimu mąstyti
priklauso dvasios sferai. Nuo seno mąstytojai narplioja klausimą, kaip 
biologija lemia mūsų psichiką ir elgesį, ar kūnas ir siela yra viena, kaip 
jie gali egzistuoti kartu ir pan.

\section{Gyvybės psichikos evoliucija}

\label{tema:gyvybes_psichikos_evoliucija}

Filogenezėje (\gls{filogeneze}) – gyvybės psichikos evoliucijoje svarbūs 
trys etapai: 
\begin{enumerate}
  \item gyvosios gamtos atsiradimas iš negyvosios,
  \item perėjimas nuo biologinio išorinio pasaulio atspindėjimo į psichinį 
    jo atspindėjimą,
  \item perėjimas nuo gyvūno psichikos į žmogaus sąmonę.
\end{enumerate}

Pagrindiniai biologinės evoliucijos veiksniai buvo genetinis kintamumas
(mutacijos genetinėje medžiagoje) ir natūralioji atranka. Individuali 
atranka virto individualia-grupine. Grupinė atranka skatino garsinių signalų
suvokimo, įrankių gamybos ir kitų procesų tobulėjimą. Žmogus vystėsi iš 
primatų tipo protėvių natūraliomis (biologinėmis) sąlygomis, o vėliau 
prasidėjo ir kultūrinė evoliucija. Atsirado žmogus, gebantis savo 
individualią patirtį jungti su visuomenine patirtimi. Prasidėjo žmogaus
kultūrinė evoliucija, ypač paskatinusi psichikos vystymąsi.

\section{Individuali žmogaus psichikos raida}

\label{tema:psichikos_raida}

Psichinės raidos periodai, kai iš esmės keičiasi individo santykiai
su fizine ir socialine aplinka, jo reakcijos, kai vyksta lūžiai asmenybėje,
vadinami \emph{kritiniais}. Kritiniai periodai būna pereinant iš vieno 
amžiaus tarpsnio į kitą, kai formuojasi kokybiškai nauji asmenybės 
bruožai, keičiasi savęs įvertinimas ir naujų elementų įgyja „aš“ sistema.
% FIXME
Juos gali sukelti ir atskiri įvykiai (pavyzdžiui, artimo žmogaus netekimas, 
karas).

Svarbesni psichinės raidos periodai:
\begin{itemize}
  \item \emph{pirmasis} ir ryškiausias žmogaus gyvenimo periodas yra 
    pirmieji mėnesiai po gimimo. Pradeda funkcionuoti dar vaisiuje 
    susiformavę refleksai, prasideda tiesioginė sąveika su žymiai 
    kintamesne aplinka;
  \item \emph{antrasis} periodas prasideda pirmųjų metų pabaigoje. 
    Bejėgis, nuo suaugusiųjų priklausomas kūdikis mokosi pats atlikti 
    sudėtingus judesius, vaikščioti. Prasiplečia pažinimas, vystosi 
    vaizdinių bei žodžių užuomazgos;
  \item \emph{trečiasis} prasideda maždaug po dviejų metų, kurio 
    pagrindinis bruožas – kalbos išmokimas. Vaiko dėmesys jau nukreiptas ne 
    tik į išorę, bet ir į save;
  \item \emph{ketvirtuoju} periodu (septintaisiais gyvenimo metais) 
    išryškėja vaiko „vidinė pozicija“: vaikas pradeda suvokti save kaip 
    socialinį individą;
  \item \emph{penktasis} periodas – lytinio brendimo laikotarpis 
    (12-17 metai). Intensyviai vystosi saviraiška ir 
    savirealizacija. Paaugliai jau suvokia save kaip asmenybę, kurią lygina 
    su susikurtu idealu. Jie aktyviai ieško savo vietos visuomenėje ir 
    suvokia gyvenimo prasmę, kuria ateities planus;
  \item \emph{šeštasis} periodas – brandos metas. Perėjimas į jį nėra 
    ryškus, nors jis susijęs su daug svarbių asmeninio gyvenimo įvykių 
    (sukuriama šeima, pradedama dirbti, gimsta vaikai ir t.t.);
  \item \emph{septintasis} periodas yra senėjimo pradžia: silpnėja 
    reprodukcinės funkcijos, iš šeimos pasitraukia vaikai, atsiranda 
    sunkumų profesinėje veikloje, sustiprėja vienišumo tendencijos, žmogus 
    darosi irzlesnis;
  \item \emph{aštuntasis} periodas – senatvė. Žmogus priverstas palikti 
    profesinį darbą, suyra šeima (dėl sutuoktinio mirties), susilpnėja 
    socialiniai ryšiai, dažniau galvojama apie gyvenimo pabaigą.
\end{itemize}

Čia pateiktos periodų charakteristikos, be abejo, turi daug išimčių bei yra 
nepilnos, nes skiriasi žmonių brendimo bei senėjimo tempai ir pobūdis.

\section{Psichologijos metodai ir jų praktinis taikymas}

\label{tema:psichologijos_metodai}

\subsection{Empirinių duomenų rinkimo ir matavimo būdai}

Empiriniai (patyrimu pagrįsti) faktai yra informacija apie individo elgseną
įvairiose situacijose. Stebint natūralią elgseną yra naudojamas 
\emph{išprovokuotas stebėjimas}, kai stebėtojas sudaro situacijas, panašias 
į natūralias. Šis būdas dažnai taikomas stebint socialinę elgseną. Dar 
vienas natūralios elgsenos stebėjimo būdų – \emph{stebėjimas dalyvaujant}, 
kai stebėtojas bando pasidaryti grupės nariu ir stebėti elgseną „iš grupės 
vidaus“.

Stebint dažnai siekiama apriboti elgseną veikiančių faktorių kiekį, 
praktikuojami stebėjimai specialiomis arba kontroliuojamomis sąlygomis. 
Tam taikomos įvairios užduotys, kurios turėtų nukreipti tiriamojo veiklą 
ir sudaryti sąlygas stebėti tik tam tikrą su užduotimi susijusią elgsenos 
rūšį.

Be tiriamųjų stebėsenos, taip pat taikoma ir jų \emph{apklausa}, kai 
stebėtojas pateikia klausimus tiriamajam. Klausimai būna pateikiami 
įvairiomis formomis: atvirais ar uždarais anketų klausimais, pokalbio ar 
interviu metu ir kt.

Apibendrinant, pagrindiniai duomenų rinkimo būdai yra:
\begin{description}
  \item[stebėjimas] – žmogaus išorinio elgesio fiksavimas. Tikslingas tam
    tikro reiškinio suvokimas, nereikalaujantis kištis į stebimą
    reiškinį. Stebėjimas gali būti:
    \begin{itemize}
      \item natūraliomis sąlygomis;
      \item laboratorinėmis (kontroliuojamomis) sąlygomis.
    \end{itemize}
    Stebėjimą galima skirstyti į:
    \begin{itemize}
      \item stebėjimą iš šalies;
      \item stebėjimą dalyvaujant;
      \item stebėjimą išprovokuotoje situacijoje;
      \item dienoraštinį stebėjimą;
    \end{itemize}
  \item[apklausa] gali būti raštu (anketos, klausimynai) ir
    žodžiu (struktūrizuoti / pusiau struktūrizuoti / nestruktūrizuoti
    interviu);
  \item[testavimas] – standartizuotų užduočių atlikimas, pagal kurių
    rezultatus nustatomas individo psichikos funkcijų, savybių lygis ir
    būklė.
\end{description}

Duomenų rinkimo būdo vertę nusako:
\begin{description}
  \item[patikimumas] – ar pakartojus tyrimą gautume tokius pat rezultatus;
  \item[validumas] – ar tikrai matuojama tai, ką ketinama matuoti;
  \item[jautrumas] – ar naudojamas metodas gali išaiškinti minimalius
    mane dominančio reiškinio pokyčius.
\end{description}

\subsection{Stebėjimo (aprašomoji) strategija}

\emph{Stebėjimo strategija} – tai toks mokslinio tyrimo organizavimas, kai 
nevarijuojant sąlygų siekiama gauti objektyvių faktų apie žmogaus ar gyvūno 
elgseną. Atliekant tokio pobūdžio tyrimą, gali būti taikomi visi empirinių 
duomenų rinkimo būdai. Dažnai duomenys tokiems tyrimams gaunami stebint 
natūralią elgseną. Kadangi, taikant šią strategiją, empiriniai duomenys 
dažnai renkami nekontroliuojant veiksnių ir sąlygų, galinčių turėti įtakos 
registruojamai elgsenai, tai tyrėjas negali paaiškinti, kas lėmė jo 
užregistruotų faktų ypatumus. Todėl pagal daromų išvadų pobūdį ši 
strategija vadinama aprašomąja. Yra laikomasi nuomonės, kad stebėjimai 
yra pirmoji tiek mokslo apskritai, tiek konkretaus tyrimo fazė, parengianti 
pamatą eksperimentiniams tyrimams.

\subsection{Koreliaciniai tyrimai}

Koreliaciniais tyrimais siekiama įvertinti ryšį tarp dviejų ar daugiau 
veiksnių (kintamųjų). Hipotezė apie reiškinių ryšį gali būti patikrinta ir 
įvertinta koreliaciniu tyrimu. 

Ši strategija reikalauja, kad: 
\begin{enumerate}
  \item tie reiškiniai, arba veiksniai, būtų įvertinti kiekybiškai;
  \item tie kiekybiniai įvertinimai būtų gauti tiriant tuos pačius 
    individus.
\end{enumerate}

Šis tyrimas parodo tik ryšio tarp reiškinių buvimą, tačiau neparodo,
kuris reiškinys yra priežastis, o kuris pasekmė.

\subsection{Eksperimentinė strategija}

Pagrindinis šios mokslinių tyrimų strategijos ypatumas yra tas, kad tyrėjas 
sukelia norimą tirti reiškinį ir, tikslingai keisdamas tam tikras sąlygas, 
stebi, kaip priklausomai nuo jo veiksmų kinta kiti su tuo susiję reiškiniai.
Dauguma eksperimentų dabar atliekama specialiose psichologijos 
laboratorijose, kur yra geriausios sąlygos kontroliuoti kintamuosius. Tokie 
eksperimentai vadinami \emph{laboratoriniais}. Tačiau gali būti 
eksperimentuojama ir natūralioje individo aplinkoje. Tokie eksperimentai 
vadinami \emph{natūraliaisiais}.

\subsection{Praktinis taikymas}

% FIXME Suprasti ir perdaryti normaliai.
Psichologijos metodai yra taikomi:
\begin{itemize}
  \item konsultacijose;
  \item psichoterapijoje;
  \item gydant priklausomybes.
\end{itemize}
