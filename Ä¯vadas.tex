\chapter{Įvadas}

\Gls{psichologija} nagrinėja tris pagrindines psichinių reiškinių grupes:

\begin{enumerate}
  \item psichiniai procesai (\ref{psichiniai_procesai} skyrius):
    \begin{itemize}
      \item \glspl{pojutis},
      \item suvokimas (ang. perception),
      \item atmintis,
      \item mąstymas,
      \item vaizduotė;
    \end{itemize}
  \item psichinės būsenos (\ref{psichines_busenos} skyrius):
    \begin{itemize}
      \item dėmesys,
      \item emocinė aktyvacija;
    \end{itemize}
  \item TODO: individualios asmenybės savybės:
    \begin{itemize}
      \item psichinių procesų ir būsenų individualios variacijos,
      \item „grynai“ individualios savybės:
        \begin{enumerate}
          \item temperamentas,
          \item gebėjimas,
          \item charakteris.
        \end{enumerate}
    \end{itemize}
\end{enumerate}
