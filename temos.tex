\chapter{Temos}

\begin{enumerate}
  \item \label{tema_01} (Žr. \ref{tema:psichologijos_mokslas})
    Psichologija ir jos vieta mokslų sistemoje.
  \item \label{tema_02} Psichologijos raida pasaulyje ir Lietuvoje.
  \item \label{tema_03} (Žr. \ref{tema:psichologijos_sakos})
  Šiuolaikinės psichologijos šakos.
  \item \label{tema_04} (Žr. \ref{tema:psichikos_samprata})
    Psichikos samprata ir psichinių reiškinių rūšys.
  \item \label{tema_05} (Žr. \ref{tema:kuno_sielos_santykis})
  Kūno ir sielos santykis.
  \item \label{tema_06} Biologinis elgesio pagrindas.
  \item \label{tema_07} (Žr. \ref{tema:gyvybes_psichikos_evoliucija})
  Gyvybės psichikos evoliucija.
  \item \label{tema_10} (Žr. \ref{tema:psichikos_raida})
  Individuali žmogaus psichikos raida.
  \item \label{tema_11} (Žr. \ref{tema:psichologijos_metodai})
  Psichologijos metodai ir jų praktinis taikymas.
  \item \label{tema_12} (Žr. \ref{tema:psichiniai_procesai})
    Bendra psichinių procesų charakteristika.
  \item \label{tema_13} (Žr. \ref{tema:pojuciai})
    Pojūčiai. Bendrieji jų dėsniai ir savybės.
  \item \label{tema_14} (Žr. \ref{tema:suvokimas})
    Suvokimas. Jo savybės ir bendrieji dėsniai.
  \item \label{tema_15} (Žr. \ref{tema:mastymas})
    Mąstymas. Jo savybės ir rūšys.
  \item \label{tema_16} (Žr. \ref{tema:problemu_sprendimas})
    Psichologinės sprendimo priėmimo savybės.
  \item \label{tema_17} (Žr. \ref{tema:kalba_ir_kalbejimas})
  Kalba ir kalbėjimas. Jų funkcijos ir rūšys.
  \item \label{tema_20} Individuali ir evoliucinė kalbos raida.
  \item \label{tema_21} (Žr. \ref{tema:atmintis})
    Atmintis. Jos funkcijos ir rūšys.
  \item \label{tema_22} (Žr. \ref{tema:atminties_procesai})
    Atminties procesai ir jų savybės.
  \item \label{tema_23} Individuali atminties raida ir lavinimas.
  \item \label{tema_24} (Žr. \ref{tema:mokymas_mokymasis})
  Mokymas, mokymasis ir išmokimas.
  \item \label{tema_25} (Žr. \ref{tema:igudziai})
  Įgūdžiai, jų sąveika ir įpročiai.
  \item \label{tema_26} (Žr. \ref{tema:demesys})
    Dėmesys, jo samprata ir rūšys.
  \item \label{tema_27} (Žr. \ref{tema:demesio_savybes})
    Pagrindinės dėmesio savybės ir raida.
  \item \label{tema_30} (Žr. \ref{tema:asmenybe})
  Individo ir asmenybės samprata.
  \item \label{tema_31} (Žr. \ref{tema:asmenybe}) 
  Asmenybės aktyvumas, jos struktūra ir raida.
  \item \label{tema_32} (Žr. \ref{tema:asmenybe})
  Elgesio priežastys ir jo motyvacija.
  \item \label{tema_33} Fizinis, psichinis ir socialinis aktyvumas.
  \item \label{tema_34} (Žr. \ref{tema:veiklos_rusys})
  Pagrindinės žmogaus veiklos rūšys ir jų raida.
  \item \label{tema_35} (Žr. \ref{tema:emocijos})
  Emocijų bei jausmų samprata ir jų savybės.
  \item \label{tema_36} (Žr. \ref{tema:emocijos})
  Emocijų išraiška ir jų rūšys.
  \item \label{tema_37} (Žr. \ref{tema:emocijos})
  Emocijų funkcijos ir teorijos.
  \item \label{tema_40} (Žr. \ref{tema:asmenybe})
  Asmenybė ir jos savitumo priežastys.
  \item \label{tema_41} (Žr. \ref{tema:asmenybe})
  Asmenybės tipologijos.
  \item \label{tema_42} Biologiniai ir socialiniai asmenybės komponentai.
  \item \label{tema_43} Bendravimas, jo emociniai ir 
    informaciniai komponentai.
  \item \label{tema_44} (Žr. \ref{tema:konfliktai})
  Tarpasmeniniai konfliktai ir jų sprendimas.
  \item \label{tema_45} (Žr. \ref{tema:grupe})
  Grupė, jos samprata, rūšys ir savybės.
  \item \label{tema_46} (Žr. \ref{tema:grupe})
  Grupės struktūra ir jos nario statusas.
\end{enumerate}
