\newglossaryentry{psichologija}{
  name=psichologija,
  description={
  \begin{inparaenum}[\itshape a\upshape)]
    \item mokslas, tiriantis psichikos faktus, nustatantis dėsnius ir 
      kuriantis jų aiškinimo teorijas
    \item mokslas apie žmogaus elgesį ir psichikos procesus (suvokimą, 
      mąstymą, jausmus, įvairius asmeninius išgyvenimus, kurie elgesyje 
      tiesiogiai nepasireiškia)
  \end{inparaenum}
  },
  sort=psichologija,
  plural=psichologijos
  }

\newglossaryentry{elgesys}{
  name=elgesys,
  description={viskas, ką daro organizmas},
  sort=elgesys,
  plural=elgesiai
  }

\newglossaryentry{psichikos_procesai}{
  name={psichikos procesai},
  description={vidiniai subjektyvūs patyrimai},
  sort={psichikos procesai}
  }

\newglossaryentry{centrine_nervu_sistema}{
  name={centrinė nervų sistema},
  description={galvos ir nugaros smegenys},
  sort={centrine nervu sistema}
  }

\newglossaryentry{samone}{
  name={sąmonė},
  description={savo minčių, jausmų, suvokimo žinojimas},
  sort={samone},
  }

\newglossaryentry{pojutis}{
  name={pojūtis},
  description={ 
    psichinis procesas, atsirandantis materialiems dirgikliams betarpiškai 
    veikiant mūsų jutimo organus
    },
  sort={pojutis},
  plural=pojūčiai
  }

\newglossaryentry{kinestezija}{
  name={kinestezija},
  description={
    kūno, jo dalių padėties ir judėjimo jutimas nežiūrint į juos (DLKŽ)
    },
  sort={kinestezija}
  }

\newglossaryentry{objektyvioji_psichofizika}{
  name={objektyvioji psichofizika},
  description={
    tada, kai pojūtis turi fizikinį atitikmenį (pavyzdžiui, garsumas, 
    šviesumas)
    },
  sort={objektyvioji psichofizika}
  }

\newglossaryentry{subjektyvioji_psichofizika}{
  name={subjektyvioji psichofizika},
  description={
    tada, kai pojūtis neturi fizikinio atitikmens (pavyzdžiui, laimė)
    },
  sort={subjektyvioji psichofizika}
  }

\newglossaryentry{apatinis_absoliutinis_pojucio_slenkstis}{
  name={apatinis absoliutinis pojūčio slenkstis},
  description={
    pojūčio stiprumo riba, kurios žemiau esantis pojūčio stiprumas yra 
    nebejuntamas
    },
  sort={apatinis absoliutinis pojūčio slenkstis}
  }

\newglossaryentry{virsutinis_absoliutinis_pojucio_slenkstis}{
  name={viršutinis absoliutinis pojūčio slenkstis},
  description={
    pojūčio stiprumo riba, kurios aukščiau esantis pojūčio stiprumas yra
    nebejuntamas (juntamas tiesiog skausmas)
    },
  name={viršutinis absoliutinis pojūčio slenkstis}
  }

\newglossaryentry{pojuciu_skyrimo_slenkstis}{
  name={pojūčių skyrimo slenkstis},
  description={
    mažiausias dirgiklio stiprumo pokytis, kuris sukelia vos juntamą pojūčio
    stiprumo pokytį
    },
  sort={pojūčių skyrimo slenkstis}
  }

\newglossaryentry{ikislenkstinis_dirgiklis}{
  name={ikislenkstinis dirgiklis},
  description={
    dirgiklis, veikiantis nervų sistemą, bet nesukeliantis pojūčio, nes
    jo stiprumas mažesnis už apatinį absoliutųjį pojūčio slenkstį
    },
  name={ikislenkstinis dirgiklis}
  }

\newglossaryentry{latentinis_periodas}{
  name={latentinis periodas},
  description={
    vėlavimas tarp dirgiklio ir pojūčio
    },
  sort={latentinis periodas}
  }

\newglossaryentry{ontogeneze}{
  name=ontogenezė,
  description={
    individuali organizmo raida, nuoseklių morfologinių, fiziologinių, 
    biocheminių ir funkcinių kitimų visuma nuo individo atsiradimo iki
    natūralios mirties (DLKŽ)
    },
  sort=ontogenezė
  }

\newglossaryentry{filogeneze}{
  name={filogenezė},
  description={
    gyvūnų ir žmogaus psichikos atsiradimas ir raida (DLKŽ)
    },
  sort=filogenezė
  }

\newglossaryentry{apercepcija}{
  name={apercepcija},
  description={
    suvokimas, priklausomas nuo ankstesnės žmogaus patirties, psichinės 
    veiklos ir individualių savybių (DLKŽ)
    },
  sort=apercepcija
  }

\newglossaryentry{pagava}{
  name={pagava},
  description={suvokimas (DLKŽ)},
  sort=pagava
  }
  
\newglossaryentry{atmintis}{
  name={atmintis},
  description={
    savybė kaupti, saugoti ir panaudoti asmeninę ir visuomeninę patirtį
    },
  sort=atmintis
  }

\newglossaryentry{atminties_subjektas}{
  name={atminties subjektas},
  description={
    tas, kuris įsimena
    },
  sort={atminties subjektas}
  }

\newglossaryentry{atminties_objektas}{
  name={atminties objektas},
  description={
    tas, kurį įsimena
    },
  sort={atminties objektas}
  }

\newglossaryentry{atgaminimas}{
  name={atgaminimas},
  description={
    informacijos įkėlimas į operatyviąją atmintį ir jos aktualizacija
    },
  sort=atgaminimas
  }

\newglossaryentry{pedsaku_konsolidacijos_laikas}{
  name={pėdsakų konsolidacijos laikas},
  description={
    laiko tarpas po įsiminimo, kai informacijos neįmanoma atgaminti
    },
  sort={pėdsakų konsolidacijos laikas}
  }

\newglossaryentry{problemine_situacija}{
  name={probleminė situacija},
  description={
    situacija, kur seni sprendimo metodai neveikia
    },
  sort={probleminė situacija}
  }

\newglossaryentry{indukcija}{
  name={indukcija},
  description={
    išvadų gavimas iš atskirų teiginių (DLKŽ)
    },
  sort=indukcija
  }

\newglossaryentry{dedukcija}{
  name=dedukcija,
  description={
    vieno sprendimo išvedimas iš kito pagal logikos dėsnius (DLKŽ)
    },
  sort=dedukcija
  }

\newglossaryentry{demesys}{
  name=dėmesys,
  description={
    psichinės veiklos nukreipimas ir sukoncentravimas į kokį nors objektą
    (objektus)
    },
  sort=dėmesys
  }

\newglossaryentry{astenija}{
  name=astenija,
  description={
    padidinto jautrumo, dirglumo, greito išsekimo ir nuovargio derinys;
    pasitako įgimta dėl menko fizinio sudėjimo, arba ją sukelia persirgtos
    sunkios ligos (DLKŽ)
    },
  sort=astenija
  }

\newglossaryentry{frustracija}{
  name=frustracija,
  description={
    nemaloni įtempta psichinė būsena, kurią sukelia objektyviai neįveikiami
    arba įsivaizduojami sunkumai, trukdantys pasiekti tikslą, patenkinti
    poreikį (DLKŽ)
    },
  sort=frustracija
  }

\newglossaryentry{regresija}{
  name=regresija,
  description={
    psichinės gynybos mechanizmus: grįžimas prie nebrandaus elgesio; žmogus
    regresija nesąmoningai naudojasi tada, kai neįveikia išorinių ar
    vidinių konfliktų ir nori priversti kitus išspręsti savo gyvenimo 
    klausimus (DLKŽ)
    },
  sort=regresija
  }

\newglossaryentry{introspekcija}{
  name=introspekcija,
  description={
    [lot. introspecto – žiūriu į vidų] pagrindinis introspekcinės 
    psichologijos metodas: specialaus pasirengimo ir lavinimosi reikalingas
    savo sąmonės būsenų ir procesų stebėjimas bei aprašymas (DLKŽ)
    },
  sort=introspekcija
  }

\newglossaryentry{gerontopsichologija}{
  name=gerontopsichologija,
  description={
    psichologijos šaka, tirianti pagyvenusių ir senyvų žmonių psichikos bei
    elgesio dinamiką ir dėsningumus (DLKŽ)
    },
  sort=gerontopsichologija
  }

\newglossaryentry{demesio_savybe}{
  name={dėmesio savybė},
  description={
    bendriausias dėmesio bruožas įvairiose psichinės veiklos srityse
    },
  sort={dėmesio savybė}
  }

\newglossaryentry{suvokimas}{
  name={suvokimas (percepcija)},
  description={
    daikto ar reiškinio visumos atspindėjimas, jam tiesiogiai veikiant
    jutimo organus
    },
  sort=suvokimas
  }

\newglossaryentry{suvokinys}{
  name=suvokinys,
  description={
    sąmonėje susiformavęs daikto, reiškinio ar įvykio visumos atspindys
    },
  sort=suvokinys
  }

\newglossaryentry{mastymas}{
  name=mąstymas,
  description={
    apibendrintas ir netiesioginis tikrovės atspindėjimo sąmonėje procesas,
    leidžiantis pažinti ne tik tikrovės daiktus, įvykius ir kitus
    reiškinius, bet ir jų santykius bei priežastinius ryšius
    },
  sort=mąstymas
  }

\newglossaryentry{geneze}{
  name=genezė,
  description={
    atsiradimo istorija, kilmė (DLKŽ)
    },
  sort=genezė
  }

\newglossaryentry{emocija}{
  name=emocija,
  description={
    siaurąja prasme tai yra trumpalaikis, situacinis išgyvenimas, kilęs dėl
    biologinių organizmo poreikių (alkio, troškulio ir kt.) bei nesudėtingų
    išorinės aplinkos poveikių (malonus ar nemalonus kvapas, šaižus garsas,
    skausmas ir t.t.)
    },
  sort=emocija
  }

\newglossaryentry{jausmas}{
  name=jausmas,
  description={
    palyginus su emocijomis pastovūs ir ilgalaikiai žmonių išgyvenimai,
    atspindintys asmenybės, kaip visuomenės nario, santykį su aplinka
    },
  sort=jausmas
  }

\newglossaryentry{temperamentas}{
  name=temperamentas,
  description={
    pastovios asmenybės savybės, pasireiškiančios psichinių reiškinių
    intensyvumu, tempais ir pastovumu.
    },
  sort=temperamentas
  }

\newglossaryentry{ekstraversija}{
  name=ekstraversija,
  description={
    asmenybės ypatybė, kuri reiškiasi didesniu dėmesiu aplinkai, negu sau,
    poreikiu veikti, bendrauti su žmonėmis
    },
  sort=ekstraversija
  }

\newglossaryentry{ekstravertas}{
  name=ekstravertas,
  description={
    asmenybės tipas: asmuo, linkęs gyvai reaguoti į aplinką, sugebantis
    greitai užmegzti ryšius su kitais žmonėmis, lengvai reikšti mintis ir
    jausmus
    },
  sort=ekstravertas
  }

\newglossaryentry{charakteris}{
  name=charakteris,
  description={
    tokios individualios asmenybės savybės, kurios pasirodo tos
    asmenybės tom savybėm tipiškose situacijose
    },
  sort=charakteris
  }

\newglossaryentry{gebejimai}{
  name=gebėjimai,
  description={
    individualios asmenybės savybės, nuo kurių priklauso žinių, mokėjimų ir 
    įgūdžių susiformavimo greitis
    },
  sort=gebėjimai
  }

\newglossaryentry{individas}{
  name=individas,
  description={
    homo sapiens
    },
  sort=individas
  }

\newglossaryentry{asmenybe}{
  name=asmenybė,
  description={
    individas, įjungtas į socialinius santykius
    },
  sort=asmenybė
  }

\newglossaryentry{aspiraciju_lygis}{
  name={aspiracijų lygis},
  description={
    asmenybės sau keliamų tikslų lygis
    },
  sort={aspiracijų lygis}
  }

\newglossaryentry{afazija}{
  name={afazija},
  description={
    nebylumas, nesugebėjimas kalbėti ir suprasti kalbos; afazija susijusi
    su galvos smegenų žievės kairiojo pusrutulio tam tikrų centrų pažeidimu;
    kalbos ir klausos organai būna nepakitę (DLKŽ)
    },
  sort=afazija,
  plural=afazijos
  }

\newglossaryentry{psichika}{
  name=psichika,
  description={
    smegenų funkcija atspindėti aplinką ir reguliuoti organizmo sąveiką su
    ja (DLKŽ)
    },
  sort=psichika
  }

\newglossaryentry{psichiniai_procesai}{
  name={psichiniai procesai},
  description={
    dinamiški, dažnai besikeičiantys reiškiniai, kurie prasideda veikiant
    išoriniams ar vidiniams dirginimams ir baigiasi jiem nutrūkus
    },
  sort={psichiniai procesai}
  }
