\chapter{Psichiniai procesai}

\label{tema:psichiniai_procesai}

\Gls{psichiniai_procesai} – \glsentrydesc{psichiniai_procesai}. Jiems yra 
būdinga tai, jog jie duoda vidaus ir išorės atspindį. Psichinių procesų
raida:
\begin{enumerate}
  \item \gls{pojutis};
  \item \gls{suvokimas};
  \item \gls{mastymas} ir vaizduotė.
\end{enumerate}

\section{Pojūčiai}

\label{tema:pojuciai}

\Gls{pojutis} – \glsentrydesc{pojutis}. Kiekvienas jutimo organas
susideda iš trijų pagrindinių blokų:
\begin{enumerate}
  \item recepcinis blokas – daviklių, kurie atlieka transformavimo funkciją 
    (dirginimo energijos į nervinio jaudinimo (elektros) energiją), blokas;
  \item perdavimo blokas;
  \item centrinės nervų sistemos (\gls{centrine_nervu_sistema}) blokas.
\end{enumerate}

\Glspl{pojutis} skirstomi į tipus: 
\begin{description}
  \item[eksterorecepciniai pojūčiai] – atsiranda dėl išorinių dirgiklių
    poveikio kūno paviršiuje esantiems receptoriams:
    \begin{description}
      \item[nuotoliniai (distanciniai)] – rega, klausa;
      \item[sąlytiniai (kontaktiniai)] – lytėjimas, skonis;
      \item[tarpiniai] – uoslė;
    \end{description}
  \item[interorecepciniai (organiniai) pojūčiai] – juntama vidaus organų
    veikla, fiziologinės būsenos:
    \begin{itemize}
      \item alkis,
      \item troškulys,
      \item miego poreikis,
      \item deguonies poreikis;
    \end{itemize}
  \item[propriorecepciniai (\gls{kinestezija}) pojūčiai] – juntamas kūno
    dalių judėjimas ir jų tarpusavio padėtis. Veikia raumenyse, sausgyslėse
    ir sąnariuose esantys specialūs (propriorecepciniai) receptoriai.
\end{description}

Bendrieji pojūčių dėsniai:
\begin{description}
  \item[pojūčio kokybė] – atspindi materijos judėjimo formos (pavyzdžiui,
    garso tembras);
  \item[pojūčio kiekybė (amplitudinė charakteristika)] – kaip išmatuoti
    pojūčio intensyvumą ir kokiu adekvačiu metodu tai patikrinti;
  \item[pojūčio trukmė] – laikinė charakteristika (dėl pojūčių „užtempimų“
    atsiranda iliuzijos, nes įvairus jutimai turi skirtingus latentinius
    periodus);
  \item[pojūčių adaptacija] – pojūčių intensyvumo kitimas. Rūšys:
    \begin{enumerate}
      \item \label{pojucio_bukimas} pojūčio bukimas (mažėjimas veikiant 
        stipriam dirgikliui);
      \item pojūčio stiprėjimas (žymiai lėtesnis nei \ref{pojucio_bukimas},
        vyksta veikiant silpnam dirgikliui);
      \item visiškas pojūčio išjungimas veikiant vidutinio intensyvumo
        dirgikliui;
    \end{enumerate}
  \item[sensibilizacija] – dirgiklio sukeltas nervinių centrų jautrumo
    padidėjimas veikiant vidinėms sąlygoms (pavyzdžiui, užsiimant 
    specialiomis treniruotėmis, arba dėl bendrų fiziologinių pakitimų
    organizme – moterims nėštumo metu);
  \item[sinestezija] – nevalingas dviejų ar daugiau pojūčių atsiradimas,
    dirginant tik vieno iš tų pojūčių jutimo organą. (Labiausiai 
    paplitusi sinestezija – spalvinė klausa: garsas sukelia ne tik
    klausos, bet ir spalvos pojūtį.)
\end{description}

Susiję terminai:
\begin{description}
  \item[\gls{objektyvioji_psichofizika}] 
    \glsentrydesc{objektyvioji_psichofizika};
  \item[\gls{subjektyvioji_psichofizika}] 
    \glsentrydesc{subjektyvioji_psichofizika};
  \item[\gls{apatinis_absoliutinis_pojucio_slenkstis}] 
    \glsentrydesc{apatinis_absoliutinis_pojucio_slenkstis};
  \item[\gls{virsutinis_absoliutinis_pojucio_slenkstis}] 
    \glsentrydesc{virsutinis_absoliutinis_pojucio_slenkstis};
  \item[\gls{pojuciu_skyrimo_slenkstis}] 
    \glsentrydesc{pojuciu_skyrimo_slenkstis};
  \item[\gls{latentinis_periodas}] 
    \glsentrydesc{latentinis_periodas};
  \item[\gls{weber-fechner}] 
    \glsentrydesc{weber-fechner}.
\end{description}

\section{Suvokimas}

\label{tema:suvokimas}

\Gls{suvokimas} – \glsentrydesc{suvokimas}. Kaip atskirti pojūtį nuo 
suvokimo? Pojūtis duoda atskirą savybę, o suvokimas vaizdą. Taip pat 
suvokimas yra susijęs su tuo, kas yra saugoma ilgalaikėje atmintyje.
Suvokimui didelę įtaką turi individo turima patirtis, emocinė būsena.
Suvokimas, priklausomas nuo ankstesnės žmogaus patirties, psichinės
veiklos ir individualių savybių dar vadinamas apercèpcija.

Bendrieji suvokimo (pagavos), kaip \emph{atspindžio}, dėsniai:
\begin{enumerate}
  \item \emph{daiktiškas} (žmogus gali suvokti daiktų realumą ir 
    lokalizuoti juos erdvėje);
  \item \emph{įprasmintas} (suvokiamus objektus įtraukiame į tam tikrą 
    sistemą).
  \item \emph{vientisas} (savybė atskirianti pojūtį nuo suvokimo; dėl
    suvokimo vientisumo suvoktas vaizdas yra visada pilnas – jei ko nors
    trūksta, tai yra prikuriama);
  \item \emph{struktūriškas} (tarp visumos dalių yra struktūriniai 
    santykiai, kurie lemia objekto suvokimą, priskyrimą tam tikrai klasei);
  \item \emph{konstantiškas} (stalas, kad ir iš kurios pusės bežiūrėtum
    yra suvokiamas vienodai; tai leidžia žmonėms suvokti daiktus, kaip 
    sąlyginai pastovius).
\end{enumerate}
Visos šios savybės yra įgytos, o ne įgimtos, nes jų atsiradimui yra
būtina organizmo aktyvi veikla išorinio pasaulio atžvilgiu. Taigi, nors
suvokinys yra atspindys (daiktas), bet tam, kad žmogus galėtų jį
susidaryti, reikia, jog žmogus atliktų percepcinius veiksmus. Pavyzdžiui,
žmogaus akys nuolat atlieka mikrojudesius stebimo objekto atžvilgiu.
Jei objektas stabiliai veikia tą pačią tinklainės vietą, tai tada 
neišeina susidaryti tikslaus jo vaizdo. Su rega susiję keletas svarbių 
sąvokų:
\begin{description}
  \item[sekimo judesiai] – akių judesiai įgalinantys stebėti judančius
    daiktus;
  \item[sakadiniai judesiai] – staigūs akies „šuoliai“, kurie būna skaitant
    knygą arba žiūrint į paveikslą;
  \item[objekto fiksavimas] – žvilgsnio stabilus nukreipimas į objektą.
    Fiksavimo metu apie objektą gaunama daugiausiai informacijos, bet
    net ir jo metu vyksta akių mikrojudesiai.
\end{description}
% Vadovėlyje 142 psl.

\subsection{Suvokimo rūšys}

Galima išskirti erdvės, laiko ir judesio suvokimus.

\subsubsection{Erdvės suvokimas}

Suvokiami objektai yra erdvėje: turi plotį, aukštį ir ilgį, yra reljefiški.
Daikto projekcija tinklainėje arba jo liečiamas odos paviršius turi tik du 
parametrus, bet, funkcionuojant visai sensorinei sistemai, veikiant ir 
kitiem pojūčiams, suvokiamas ir atstumas iki to daikto. Normalaus žmogaus 
regėjimas yra \emph{binokulinis} (abiakis). Žiūrint į toli esantį objektą,
akių regimosios ašys beveik lygiagrečios. Kuo suvokiamas objektas arčiau, 
tuo ašių sudaromas kampas didesnis. Konvergaciniai akių judesiai atsispindi 
smegenų žievėje dirginimais, signalizuojančiais apie suvokiamo daikto dydį
ir atstumą iki jo.

Atstumą padeda suvokti ir \emph{binokulinis paralaksas} (nukrypimas). 
Objekto vaizdas abiejose akių tinklainėse atsispindi truputį skirtingai. 
Suvokiant daiktą abu šie vaizdai susilieja į vieną bei taip gaunamas
daikto reljefiškumas.

Suvokiant daiktą erdvėje yra naudojama rega, lyta ir \gls{kinestezija}
bei yra nagrinėjami tokie parametrai:
\begin{enumerate}
  \item forma;
  \item nuotolis;
  \item kryptis;
  \item kontūras;
  \item laikas.
\end{enumerate}

\subsubsection{Laiko suvokimas}

\emph{Laiko suvokimas} yra tikrovės reiškinių trukmės, greičio ir nuoseklumo
atspindėjimas, padedantis žmogui orientuotis gamtos ir visuomenės gyvenimo
įvykiuose. Gyvūnams orientuotis laike padeda vadinamasis „biologinis 
laikrodis“. Jis gali būti įgimtas arba įgytas. Žmogus laiką suvokia ir 
vertina nepasikliaudamas vien sensoriniu „biologiniu laikrodžiu“, bet
remiasi aukštesniais tarpiniais procesais (atmintimi, mąstymu). Informaciją
apie įvykius jis trumpiau ar ilgiau išlaiko atmintyje ir atitinkamai 
perkuria. Laiko trukmės atkūrimas priklauso nuo tuo metu gaunamos 
informacijos sudėtingumo, aiškumo. 

Subjektyviai suvokiamas laikas dažnai
būna lyginamas su objektyviai matuojamais laiko vienetais (valandomis, 
sekundėmis). Toks palyginimas įpratina žmogų realiai vaizduotis laiko 
talpumą ir planuoti darbą.

\subsubsection{Judėjimo suvokimas}

Psichologijoje \emph{judėjimu} vadiname materialių (gyvų ir negyvų) kūnų
padėties kitimą erdvėje per tam tikrą laiko tarpą. Judėjimo suvokimas 
neatsiejamai susijęs su erdvės ir laiko (greičio) suvokimu. Suvokiant 
judėjimą natūraliomis sąlygomis, paprastai laikomasi kokio nors atskaitos
taško – stacionaraus ar lėčiau judančio objekto. Pagal šį tašką nustatoma
tiek judėjimo kryptis, tiek greitis. Jutimo organai nepadeda suvokti nei 
labai lėtai, nei labai greitai judančių objektų. Tokiais atvejais judėjimas
suvokiamas tarpiniu būdu – mąstymu.

\section{Atmintis}

\label{tema:atmintis}

\Gls{atmintis} – \glsentrydesc{atmintis}.

Atminties funkcijos:

\begin{enumerate}
  \item informacijos kaupimas;
  \item informacijos saugojimas;
  \item asmeninės ir visuomeninės patirties panaudojimas.
\end{enumerate}

Žmogaus atmintį sudaro keturi procesai:

\begin{enumerate}
  \item įsiminimas;
  \item saugojimas;
  \item atgaminimas;
  \item užmiršimas.
\end{enumerate}

Yra keletas atminties teorijų. Jos skirstomos į grupes:

\begin{itemize}
  \item kibernetinės;
  \item fizikinės (pavyzdžiui, laikoma, kad atmintis yra kaip hologramos);
  \item biocheminės;
  \item neurofiziologinės (ieškoma nervinių mechanizmų);
  \item psichologinės (pagal tai ar teikia pirmumą subjektui 
    (\gls{atminties_subjektas}) ar objektui (\gls{atminties_objektas}),
    psichologijos teorijos dar skirstomos į du pogrupius).
\end{itemize}

% TODO Išsiaiškinti.
%+ Atmintis yra hierarchinė.
%+ Korsokofo sindromas – Amono rage atsiranda pasikeitimai dėl kurių žmogus
%  užmiršta net ir savo pavardę. (Būdinga alkoholikams.)

% TODO Nurodyti, kad pavyzdys. PRADŽIA
Asociatyvistinė atminties teorija (psichologinė teorija akcentuojanti
objektą) teigia, jog yra trijų tipų asociacijos:

\begin{itemize}
  \item pagal laiko ir erdvės gretutinumą;
  \item pagal panašumą;
  \item pagal kontrastą.
\end{itemize}
% PABAIGA

% TODO Nurodyti, kad pavyzdys. PRADŽIA
Geštaltpsichologija (Geštalt yra nedaloma vaizdo dalis, kuri yra įsimenama)
yra psichologinė teorija akcentuojanti subjektą.
% PABAIGA

Atminties rūšys, pagal dominuojančią psichinės veiklos rūšį:

\begin{enumerate}
  \item motorinė (judėjimo) – būdinga balerinom, šaltkalviams, dailidėm;
  \item emocinė (jausminė) – būdinga rašytojams, aktoriams;
  \item vaizdinė (susieta su suvokimu):
    \begin{enumerate}
      \item regos (būdinga dailininkams);
      \item klausos;
      \item lytos;
      \item uoslės;
      \item skonio;
    \end{enumerate}
  \item žodinė (sąvokinė) – žodžiais pateikiamos medžiagos įsiminimas, 
    saugojimas ir užmiršimas.
\end{enumerate}

Ilgalaikė atmintis yra Amono rage.

\label{tema:atminties_procesai}

Įsiminimo savybės:

\begin{itemize}
  \item išrankus – ne viskas yra įsimenama (išrankumo laipsnis 
    priklauso nuo motyvacijos ir emocinio požiūrio į įsimenamą objektą);
  \item skirtingo valingumo laipsnio:
    \begin{itemize}
      \item nevalingas – įsiminimą kontroliuoja pati įsimenamoji 
        informacija;
      \item valingas – įsiminimą kontroliuoja pats žmogus (valingam
        įsiminimui yra būtinas kartojimas);
    \end{itemize}
  \item pagal saugojimo trukmę, išskiriamos įsiminimo rūšys:
    \begin{itemize}
      \item operatyvioji – atsimenama tik tol kol reikia aptarnauti 
        konkrečią operaciją;
      \item trumpalaikė – atsimenama tol, kol informacija perkeliama iš
        operatyviosios į ilgalaikę;
      \item ilgalaikė – atsimenama iki gyvenimo galo;
    \end{itemize}
\end{itemize}

\Gls{atgaminimas} – \glsentrydesc{atgaminimas}. (Aktualizacija reiškia, kad
atgaminama informacija turi būti aktuali dabartinei veiklai.) Atgaminimas
taip pat gali būti valingas ir nevalingas.
% FIXME Aktyvumas, kryptingų asociacijų iššaukimas.

Užmiršimas yra informacijos praradimas. Jis yra atvirkštinė to paties 
išlaikymo proceso pusė. Laikui bėgant atsimenamos informacijos kiekis
mažėja, kol pasiekia $\Delta y$ – kiekį, kurį žmogus atsimena visą likusį
gyvenimą. Laikui bėgant įsiminta informacija kinta ne tik kiekybiškai, bet
ir kokybiškai – ji yra vis labiau ir labiau apibendrinama.
% FIXME Gali būti traktuojamas, kaip uždraudimas atsiminti. 
% Savybės: 1) gylis; 2) kartojimo dažnio funkcija; 3) užmirštama tik
% konkreti informacijos forma.

\section{Mąstymas}

\label{tema:mastymas}

\Gls{mastymas} – \glsentrydesc{mastymas}. Mąstymo nagrinėjama informacija 
yra abstrakti, atitrūkusi nuo savo fizinio nešėjo. Atsiranda sąvokos.
Mąstymas nuo suvokimo skiriasi tuo, kad yra idealus, o ne realus. Suvokimas
yra realus, nes jis duoda dokumentinį dabarties atvaizdį.

Mąstymas ir vaizduotė yra naudojami sprendimų probleminėse situacijose
(\gls{problemine_situacija}) paieškai. Jeigu probleminė situacija yra aiški,
tada naudojame mąstymą, o jei neaiški – vaizduotę.

\label{tema:problemu_sprendimas}

Uždavinio sprendimo fazės:
\begin{enumerate}
  \item uždavinio aptikimas;
  \item uždavinio analizė;
  \item sprendimo sumanymo kūrimas;
  \item sumanymo patikrinimas.
\end{enumerate}

Žmogus naudoja tris skirtingas sprendimo priėmimo strategijas. Sėkmė 
dažniausiai priklauso nuo sėkmingo strategijos pasirinkimo. Tos strategijos:
\begin{description}
  \item[klaidų ir bandymų] – universali, bet gali ilgai užtrukti;
  \item[algoritminė] – kito žmogaus surasto sprendimo pritaikymas;
  \item[euristinė] – nuo klaidų ir bandymų metodo skiriasi tuo, kad bandome
    ne visus variantus, o tik tuos, kurie mums atrodo (dažnai vertiname 
    patys to nesuvokdami – pasąmonės lygyje) labiausiai tikėtini.
\end{description}

Sprendimo priėmimo keliai:

\begin{description}
  \item[indukcinis] – nuo dalių prie bendro;
  \item[dedukcinis] – nuo bendro prie dalių.
\end{description}

Mąstymo rūšys (pagal tai, su kokiais objektais atliekamos mąstymo 
operacijos):

\begin{enumerate}
  \item vaizdinis – veiksminis (vaikas ardo savo žaislus norėdamas 
    išsiaiškinti jų struktūrą);
  \item verbalinis, prasminis, kalbinis;
  \item abstraktusis, švarusis – mąstymas „be daiktų“ (sąvokinis, 
    matematinis).
\end{enumerate}

Individualios mąstymo savybės:

\begin{itemize}
  \item mąstymo savarankiškumas – juo jis aukštesnis, tuo žmogus yra
    labiau savarankiškas, jam mažesnę įtaką daro išorės primetamos
    mąstymo normos; šioje vietoje iškyla problema, kad labai aukštą
    mąstymo savarankiškumą turintys žmonės gali nepritapti visuomenėje
    – tapti jos atžvilgiu nusikaltėliais;
  \item mąstymo lankstumas – sugebėjimas greitai pakeisti mąstymo
    strategiją pasikeitus situacijai;
  \item mąstymo greitis;
  \item esmės išskyrimas – sugebėjimas apibrėžti probleminę situaciją 
    vadovaujantis esminėmis savybėmis.
\end{itemize}

\section{Vaizduotė}

% FIXME Pastaba.
Amerikietiškoji tradicija vaizduotės, kaip atskiros neskiria. Europietiškoji
skiria, nors tas skyrimas nuo mąstymo yra gan sąlyginis.

Vaizduotės rūšys:

\begin{itemize}
  \item kūrybinė vaizduotė – kai reikia atlikti intensyvią sintezę, 
    rezultatas turi būti unikalus;
  \item atkuriančioji vaizduotė. % TODO Išsiaiškinti.
\end{itemize}

Vaizduotė, kaip ir mąstymas sprendimui rasti naudoja analizę (skaidymą) ir
sintezę, bet skirtingai nuo jo jungia logiškai nesujungiamus dalykus.

Veikiant vaizduotei kinta psichika, dėl valios blokuojasi pojūčiai ir dėl
to atsiranda pokyčiai. % TODO Performuluoti.

Vaizdinių pertvarkymo vaizduotėje būdai:

\begin{description}
  \item[agliutinavimas] – skirtingų objektų, vaizdinių dalių sujungimas
    į naujas visumas; agliutinacijoje akivaizdūs vaizdinių disosiacijos
    ir asociacijos procesai;
  \item[akcentavimas] – objekto atskirų dalių pabrėžimas, išryškinimas;
  \item[hiperbolizavimas] – vaizduojamo objekto arba jo atskirų dalių
    padidinimas, sumažinimas, dalių pagausinimas;
  \item[schematizavimas] – objektų grupėms būdingų bruožų išryškinimas,
    atmetant individualias ypatybes;
  \item[įterpimas] – vaizdinių ar objekto ypatybių perkėlimas į naują 
    kontekstą, kuriame jie įgyja kitą prasmę;
  \item[tipizavimas] – sudėtingas vaizdinių pertvarkymas, išryškinant
    grupei būdingas ypatybes per atskirus individus.
\end{description}

% FIXME Įdomus faktas.
Reguliuoti kvėpavimą išmokome todėl, kad to reikalauja mūsų kultūra – mes
visi turime mokėti šnekėti.

% FIXME Įdomus faktas.
Fiziologiją galima valdyti pasitelkiant vaizduotę: valdyti kvėpavimą, 
širdies darbą ir panašiai.

\section{Įgūdžiai ir įpročiai}

\label{tema:igudziai}

TODO: Iškelti ir sujungti.

\subsection{Įgūdžiai}

\emph{Įgūdžiai} - suautomatinti sąmoningos veiklos komponentai. Jie padeda 
veikti tiksliai, greitai ir lengvai. Tai yra aukštesnė veiklos išmokimo 
pakopa negu \emph{mokėjimas}, nes veiklai atlikti nebereikia veiksmų 
sąmoningai kontroliuoti, jie būna suautomatinti. Priklausomai nuo 
vyraujančių įgūdžiuose veiksmų jie yra skirstomi į keturias grupes:
\begin{itemize}
  \item \emph{sensorinius} - įgudime pagrindinis vaidmuo tenka jutimo
  organams (pvz., raidžių atpažinimas skaitant)
  \item \emph{motorinius} - pagrindinis vaidmuo tenka raumenų veiksmams
  (pvz., pusiausvyros išlaikymas važiuojant dviračiu)
  \item \emph{sensomotorinius} - motoriniai veiksmai yra reguliuojami 
  jutimo organų (pvz., rašymas, vairavimas)
  \item \emph{intelektiniai} - dalyvauja protinėje veikloje (pvz., veiklos 
  planavimo, problemų sprendimo).
\end{itemize}

Pastebėta vienų įgūdžių įtaka kitiems. Nustatytos dvi tokios sąveikos rūšys:
\begin{itemize}
  \item \emph{įgūdžių perkėlimas} - teigiama senų įgūdžių įtaka naujai 
  besiformuojantiems. Senus galima panaudoti formuojant naujus dėl tapatingų
  veiksmų (pvz., akardeonistui lengva išmokti groti pianinu)
  \item \emph{įgūdžių interferencija} - neigiama senų įgūdžių įtaka. 
  Interferencijos priežastis gali būti perkėlimas iš vienų įgūdžių į kitus
  tokių veiksmų, kurie faktiškai yra šiek tiek skirtingi ir negali būti
  tiesiogiai perkeliami (pvz., garsų tarimas mokantis svetimos kalbos). 
  Nustatyta, jog mažiau tarpusavyje interferuoja tvirtesni įgūdžiai.
\end{itemize}

\subsection{Įpročiai}

\emph{Įpročiai} - suautomatinti veiklos komponentai susiję su poreikiu juos
kartoti. Įpročius ir įgūdžius tarpusavyje skiria šios ypatybės: įgūdžius 
panaudojame sąmoningai, o įpročių veiksmus nuolat atliekame nesąmoningai, ne
visada tobulai. Dažnai kartojami įpročių veiksmai tampa daugiau ar mažiau
pastoviomis asmenybės savybėmis. Naudingi įpročiai formuojasi nesąmoningai
mėgdžiojant kitus asmenis: tiek specialiai auklėjant, tiek per saviauklą.
Žalingi įpročiai sąmoningai slopinami.

\section{Veiklos rūšys}

\label{tema:veiklos_rusys}

TODO: Perkelti ir sujungti.

\begin{description}
  \item[Žaidimas] savitas kitų rūšių veiklos mėgdžiojimas; žaidžiant 
    pasirengiama atitinkamai veiklos rūšiai; žaidimuose realizuojami 
    vaidmenys, kurių iš tikrųjų gyvenime žmogus negali atlikti; svarbūs yra 
    patys žaidimo veiksmai, o ne rezultatas.
  \item[Mokymasis] veikla, kurios tikslas pasiruošti darbui ar kūrybai.  
    Mokantis nėra kuriami produktai, tuo mokymasis skiriasi nuo darbo ir 
    primena žaidimą. Tai ne visada patraukli veikla, kartais priverstinė, 
    tik dėl tikslo. Bendra besimokančiojo būsena dėl to panašesnė į 
    dirbančiojo negu į žaidžiančiojo.
  \item[Darbas] pagrindinė žmonių veikla, turinti tikslą gaminti 
    materialines ar dvasines vertybes, skirtas žmonių poreikiams tenkinti. 
    Darbas yra pagrindinė veiklos rūšis ne tik dėl savo rezultato. Gerai 
    organizuotas, malonus darbas tampa žmonių poreikiu, ugdo asmenybes ir 
    sudaro galimybes joms pasireikšti kolektyve.
  \item[Kūryba] veikla, kuri duoda naujų ir originalių didelės visuomeninės 
    vertės produktų. Ji skirstome į \emph{mokslinę}, \emph{meninę} ir 
    \emph{techninę}: 
    \begin{description}
      \item[mokslinė] – tai gamtos, visuomenės ir paties žmogaus raidos 
        dėsnių nustatymas;
      \item[meninė] – apibendrintų tikrovės vaizdų, sukeliančių
        estetinius jausmus, kūrimas;
      \item[techninė] – prietaisų, tobulinančių fizinę ir psichinę veiklą, 
        projektavimas ir gaminimas.
    \end{description}
\end{description}

\section{Mokymas, mokymasis ir išmokimas}

\label{tema:mokymas_mokymasis}

Žmogaus elgseną sudaro įgimti ir išmokti judesiai bei veiksmai. Tiriant 
mokymosi procesą, būtina skirti tris glaudžiai susijusius jo aspektus - 
mokymą, mokymąsi ir išmokimą. \emph{Mokymas} yra organizuotas: pedagogo veiklos 
tikslas - ko nors išmokyti mokinius (aiškinimu, demonstravimu, kontrole,
vertinimu ir kt. veiksmais). Mokinių veikla, kuria siekiama ko nors
išmokti, vadinama \emph{mokymusi}. Mokinių vidinės ir išorinės veiklos 
pasikeitimas dėl mokymo ir mokymosi vadinamas \emph{išmokimu}. Kad mokinys
ko nors išmoktų turi būti \emph{motyvacija}, \emph{pratybos} ir 
\emph{pastiprinimas} (paskatinimai ir bausmės).
