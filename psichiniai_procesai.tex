\chapter{Psichiniai procesai}

\label{psichiniai_procesai}

% TODO Kas yra psichiniai procesai?

Psichiniams procesams yra būdinga tai, jog jie duoda vidaus ir išorės
atspindį. % FIXME Normaliai suformuluoti ir paaiškinti.

\section{Pojūčiai}

\Gls{pojutis} – \glsentrydesc{pojutis}. Kiekvienas jutimo organas
susideda iš trijų pagrindinių blokų:

\begin{enumerate}
  \item recepcinis blokas – daviklių, kurie atlieka transformavimo 
    (dirginimo energijos į nervinio jaudinimo (elektros) energiją),
    blokas;
  \item perdavimo blokas;
  \item centrinės nervų sistemos (\gls{centrine_nervu_sistema}) blokas.
\end{enumerate}

\Glspl{pojutis} skirstomi į tipus: 

\begin{description}
  \item[eksterorecepciniai pojūčiai] – atsiranda dėl išorinių dirgiklių
    poveikio kūno paviršiuje esantiems receptoriams:
    \begin{description}
      \item[nuotoliniai (distanciniai)] – rega, klausa;
      \item[sąlytiniai (kontaktiniai)] – lytėjimas, skonis;
      \item[tarpiniai] – uoslė;
    \end{description}
  \item[interorecepciniai (organiniai) pojūčiai] – juntama vidaus organų
    veikla, fiziologinės būsenos:
    \begin{itemize}
      \item alkis,
      \item troškulys,
      \item miego poreikis,
      \item deguonies poreikis;
    \end{itemize}
  \item[propriorecepciniai (\gls{kinestezija}) pojūčiai] – juntamas kūno
    dalių judėjimas ir jų tarpusavio padėtis. Veikia raumenyse, sausgyslėse
    ir sąnariuose esantys specialūs (propriorecepciniai) receptoriai.
\end{description}

Bendrieji pojūčių dėsniai:

\begin{description}
  \item[pojūčio kokybė] – atspindi materijos judėjimo formos (pavyzdžiui,
    garso tembras);
  \item[pojūčio kiekybė (amplitudinė charakteristika)] – kaip išmatuoti
    pojūčio intensyvumą ir kokiu adekvačiu metodu tai patikrinti;
  \item[pojūčio trukmė] – laikinė charakteristika;
  \item[pojūčių adaptacija] – pojūčių intensyvumo kitimas. Rūšys:
    \begin{enumerate}
      \item \label{pojucio_bukimas} pojūčio bukimas (mažėjimas);
      \item pojūčio stiprėjimas (žymiai lėtesnis nei \ref{pojucio_bukimas});
      \item visiškas pojūčio išjungimas veikiant vidutinio intensyvumo
        dirgikliui;
    \end{enumerate}
  \item[sensibilizacija] – dirgiklio sukeltas nervinių centrų jautrumo
    padidėjimas;
  \item[sinestezija] – nevalingas dviejų ar daugiau pojūčių atsiradimas,
    dirginant tik vieno iš tų pojūčių jutimo organą. (Labiausiai 
    paplitusi sinestezija – spalvinė klausa: garsas sukelia ne tik
    klausos, bet ir spalvos pojūtį.)
\end{description}

\section{Suvokimas}

\Gls{suvokimas} – \glsentrydesc{suvokimas}.

Kaip atskirti pojūtį nuo suvokimo: pojūtis duoda atskirą savybę, o 
suvokimas vaizdą. Taip pat suvokimas yra susijęs su tuo, kas yra
saugoma ilgalaikėje atmintyje.

Bendrieji suvokimo (pagavos), kaip \emph{atspindžio}, dėsniai:

\begin{enumerate}
  \item \emph{daiktiškas} (žmogus gali suvokti daiktų realumą ir 
    lokalizuoti juos erdvėje);
  \item \emph{įprasmintas} (suvokiamus objektus įtraukiame į tam tikrą 
    sistemą).
  \item \emph{vientisas} (savybė atskirianti pojūtį nuo suvokimo; dėl
    suvokimo vientisumo suvoktas vaizdas yra visada pilnas – jei ko nors
    trūksta, tai yra prikuriama);
  \item \emph{struktūriškas} (tarp visumos dalių yra struktūriniai 
    santykiai, kurie lemia objekto suvokimą, priskyrimą tam tikrai klasei);
  \item \emph{konstantiškas} (stalas, kad ir iš kurios pusės bežiūrėtum
    yra suvokiamas vienodai);
\end{enumerate}

Visos savybės yra įgytos, o ne įgimtos.

Galima išskirti erdvės, laiko ir judesio suvokimus.

Suvokimas, kaip \emph{veiksmas}: pažinimui reikalingi veiksmai 
(agnostiniai). % FIXME Ką reiškia agnostiniai veiksmai?
Tam, kad galėtume suvokti matomą objektą, mūsų akys turi jį glostyti.

\section{Atmintis}

\Gls{atmintis} – \glsentrydesc{atmintis}.

Žmogaus atmintį sudaro keturi procesai:

\begin{enumerate}
  \item įsiminimas;
  \item saugojimas;
  \item atgaminimas;
  \item užmiršimas.
\end{enumerate}

Yra keletas atminties teorijų. Jos skirstomos į grupes:

\begin{itemize}
  \item kibernetinės;
  \item fizikinės (pavyzdžiui, laikoma, kad atmintis yra kaip hologramos);
  \item biocheminės;
  \item neurofiziologinės (ieškoma nervinių mechanizmų);
  \item psichologinės (pagal tai ar teikia pirmumą subjektui 
    (\gls{atminties_subjektas}) ar objektui (\gls{atminties_objektas}),
    psichologijos teorijos dar skirstomos į du pogrupius).
\end{itemize}

% TODO Išsiaiškinti.
%+ Atmintis yra hierarchinė.
%+ Korsokofo sindromas – Amono rage atsiranda pasikeitimai dėl kurių žmogus
%  užmiršta net ir savo pavardę. (Būdinga alkoholikams.)

% TODO Nurodyti, kad pavyzdys. PRADŽIA
Asociatyvistinė atminties teorija (psichologinė teorija akcentuojanti
objektą) teigia, jog yra trijų tipų asociacijos:

\begin{itemize}
  \item pagal laiko ir erdvės gretutinumą;
  \item pagal panašumą;
  \item pagal kontrastą.
\end{itemize}
% PABAIGA

% TODO Nurodyti, kad pavyzdys. PRADŽIA
Geštaltpsichologija (Geštalt yra nedaloma vaizdo dalis, kuri yra įsimenama)
yra psichologinė teorija akcentuojanti subjektą.
% PABAIGA

Atminties rūšys, pagal dominuojančią psichinės veiklos rūšį:

\begin{enumerate}
  \item motorinė (judėjimo) – būdinga balerinom, šaltkalviams, dailidėm;
  \item emocinė (jausminė) – būdinga rašytojams, aktoriams;
  \item vaizdinė (susieta su suvokimu):
    \begin{enumerate}
      \item regos (būdinga dailininkams);
      \item klausos;
      \item lytos;
      \item uoslės;
      \item skonio;
    \end{enumerate}
  \item žodinė – žodžiais pateikiamos medžiagos įsiminimas, saugojimas ir
    užmiršimas.
\end{enumerate}

Ilgalaikė atmintis yra Amono rage.

% FIXME Suprasti ir perrašyti.
Įsiminimas yra asociatyvus. Taip pat:

\begin{itemize}
  \item išrankus – ne viskas yra įsimenama;
  \item operatyvus įsiminimas – įsiminimas, kuris reikalingas aptarnauti 
    konkrečią operaciją (po veiksmo visiškai užmirštama);
  \item trumpalaikis įsiminimas – laikoma tol, kol perkeliama iš 
    operatyviosios į ilgalaikę.
\end{itemize}

Motyvacijos nebuvimas didina įsiminimo išrankumą.

Įsiminimas gali būti valingas ir nevalingas. Nevalingą įsiminimą 
kontroliuoja informacija, o valingą pats subjektas. Valingam įsiminimui
reikalingas kartojimas.

\Gls{atgaminimas} – \glsentrydesc{atgaminimas}. (Aktualizacija reiškia, kad
atgaminama informacija turi būti aktuali dabartinei veiklai.) Atgaminimas
taip pat gali būti valingas ir nevalingas.

Užmiršimas yra informacijos praradimas. Jis yra atvirkštinė to paties 
išlaikymo proceso pusė. Laikui bėgant atsimenamos informacijos kiekis
mažėja, kol pasiekia $\Delta y$ – kiekį, kurį žmogus atsimena visą likusį
gyvenimą. Laikui bėgant įsiminta informacija kinta ne tik kiekybiškai, bet
ir kokybiškai – ji yra vis labiau ir labiau apibendrinama.

\section{Mąstymas}

\Gls{mastymas} – \glsentrydesc{mastymas}. Mąstymo nagrinėjama informacija 
yra abstrakti, atitrūkusi nuo savo fizinio nešėjo. Atsiranda sąvokos.
Mąstymas nuo suvokimo skiriasi tuo, kad yra idealus, o ne realus. Suvokimas
yra realus, nes jis duoda dokumentinį dabarties atvaizdį.

Mąstymas ir vaizduotė yra naudojami sprendimų probleminėse situacijose
(\gls{problemine_situacija}) paieškai. Jeigu probleminė situacija yra aiški,
tada naudojame mąstymą, o jei neaiški – vaizduotę.

Žmogus naudoja tris skirtingas sprendimo priėmimo strategijas. Sėkmė 
dažniausiai priklauso nuo sėkmingo strategijos pasirinkimo. Tos strategijos:

\begin{description}
  \item[klaidų ir bandymų] – universali, bet gali ilgai užtrukti;
  \item[algoritminė] – kito žmogaus surasto sprendimo pritaikymas;
  \item[euristinė] – bandoma numatyti, ne tik kas yra sprendimas, bet ir
    kaip jį galima būtų pasiekti.
\end{description}

Mąstymo rūšys (pagal tai, su kokiais objektais atliekamos mąstymo 
operacijos):

\begin{enumerate}
  \item vaizdinis – veiksminis (vaikas ardo savo žaislus norėdamas 
    išsiaiškinti jų struktūrą);
  \item verbalinis, prasminis, kalbinis;
  \item abstraktusis, švarusis – mąstymas „be daiktų“ (sąvokinis).
\end{enumerate}

Individualios mąstymo savybės:

\begin{itemize}
  \item mąstymo savarankiškumas;
  \item mąstymo lankstumas;
  \item mąstymo greitis;
  \item esmės išskyrimas – sugebėjimas apibrėžti probleminę situaciją 
    vadovaujantis esminėmis savybėmis.
\end{itemize}

\section{Vaizduotė}

% FIXME Pastaba.
Amerikietiškoji tradicija vaizduotės, kaip atskiros neskiria. Europietiškoji
skiria, nors tas skyrimas nuo mąstymo yra gan sąlyginis.

Vaizduotės rūšys:

\begin{itemize}
  \item kūrybinė vaizduotė – kai reikia atlikti intensyvią sintezę, 
    rezultatas turi būti unikalus;
  \item atkuriančioji vaizduotė. % TODO Išsiaiškinti.
\end{itemize}

Vaizduotė, kaip ir mąstymas sprendimui rasti naudoja analizę (skaidymą) ir
sintezę, bet skirtingai nuo jo jungia logiškai nesujungiamus dalykus.

Veikiant vaizduotei kinta psichika, dėl valios blokuojasi pojūčiai ir dėl
to atsiranda pokyčiai. % TODO Performuluoti.

Vaizdinių pertvarkymo vaizduotėje būdai:

\begin{description}
  \item[agliutinavimas] – skirtingų objektų, vaizdinių dalių sujungimas
    į naujas visumas; agliutinacijoje akivaizdūs vaizdinių disosiacijos
    ir asociacijos procesai;
  \item[akcentavimas] – objekto atskirų dalių pabrėžimas, išryškinimas;
  \item[hiperbolizavimas] – vaizduojamo objekto arba jo atskirų dalių
    padidinimas, sumažinimas, dalių pagausinimas;
  \item[schematizavimas] – objektų grupėms būdingų bruožų išryškinimas,
    atmetant individualias ypatybes;
  \item[įterpimas] – vaizdinių ar objekto ypatybių perkėlimas į naują 
    kontekstą, kuriame jie įgyja kitą prasmę;
  \item[tipizavimas] – sudėtingas vaizdinių pertvarkymas, išryškinant
    grupei būdingas ypatybes per atskirus individus.
\end{description}

% FIXME Įdomus faktas.
Reguliuoti kvėpavimą išmokome todėl, kad to reikalauja mūsų kultūra – mes
visi turime mokėti šnekėti.

% FIXME Įdomus faktas.
Fiziologiją galima valdyti pasitelkiant vaizduotę: valdyti kvėpavimą, 
širdies darbą ir panašiai.
