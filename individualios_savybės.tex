\chapter{Individualios savybės}

\label{tema:individualios_savybes}

\section{Temperamentas}

\label{tema:temperamentas}

Tiek žmonės, tiek gyvūnai vieni nuo kitų skiriasi pagal savo psichinės 
veiklos greitį. Šio greičio aprašymui naudojamas \gls{temperamentas}. 
Bet greitis priklauso ir nuo individo motyvacijos bei psichinės būsenos. 
Esminis skirtumas yra tai, kad temperamentas yra pastovus.

Temperamentą nulemia % FIXME Ar tikrai „nulemia“?
sumariniai (integraciniai) galvos smegenų procesai:

\begin{itemize}
  % TODO Papildyti.
  \item jaudinimas; 
  \item slopinimas.
\end{itemize}

Integracinių procesų parametrai:

\begin{itemize}
  \item amplitudė – skirtumas tarp mažiausio ir didžiausio smegenų dalies
    susijaudinimų;
  \item pusiausvyrumo laipsnis:
    \begin{itemize}
      \item pusiausvyras – visą laiką yra sujaudinta maždaug vienodo
        dydžio smegenų dalis;
      \item nepusiausvyras – galima išskirti ne tik dalių aktyvumo pokyčius,
        bet ir pačių smegenų.
    \end{itemize}
  \item paslankumas – aktyvių smegenų sričių kitimo greitis.
\end{itemize}

% FIXME Variantai, šiaip tai aštuoni – kiekvienas gali įgyti po dvi
% reikšmes, iš viso trys, tai 2^3 = 8?
Teoriškai pagal tris parametrus galima būtų išskirti šešis tipus, bet
jei nervų sistema yra silpna, tai neišeina normaliai nustatyti kitų dviejų
parametrų, todėl yra išskiriami tik keturi tipai. Jie iki šiol įvardijami
dar pagal Hipokrato terminologiją:

\begin{itemize}
  \item sangvinikas – stipri, pusiausvyra ir judri nervų sistema;
  \item cholerikas – stipri, nepusiausvyra (stiprus jaudinimas ir 
    silpnas slopinimas) nervų sistema;
  \item flegmatikas – stipri, pusiausvyra, lėta nervų sistema;
  \item melancholikas – silpna nervų sistema.
\end{itemize}

Kadangi gyvenimas priverčia žmogų į situacijas reaguoti tinkamai 
(aplinkos, kuriai priklauso žmogus, požiūriu), tai tikruosius temperamento
tipus galima būtų matyti tik pas ikimokyklinio amžiaus vaikus.

Norint nustatyti temperamento tipą yra matuojami (Plačiau –
\url{http://lt.wikipedia.org/wiki/Temperamentas}):

\begin{description}
  \item[sensityvumas (jautrumas)] Matuojamas apatinis pojūčio slenkstis;
    minimalus poreikio nepatenkinimo laipsnis, sukeliantis vos juntamą
    kentėjimą.
  \item[reaktyvumas (emocionalumas)] Matuojama, kokio stiprumo emocijas
    sukelia žmogui išorinis arba vidinis analogiško stiprumo poveikis.
  \item[aktyvumas] Jį apibūdina tai, kaip žmogus veikia, nugalėdamas 
    išorines ir vidines kliūtis.
  \item[aktyvumo ir reaktyvumo santykis] Jį parodo vyraujantis veiklos
    akstinas – nuo ko daugiau priklauso žmogaus veikla: nuo atsitiktinių
    išorinių aplinkybių (pirmauja reaktyvumas) ar nuo vidinių (pirmauja
    aktyvumas).
  \item[reakcijos tempas (ne greitis!)] Jį parodo reakcijos laikas į 
    šviesos, garso stimulus, į prasmės suvokimą.
  \item[plastiškumas – rigidiškumas] Jį parodo kaip greitai žmogus sugeba
    (plastiškumas) ar nesugeba (rigidiškumas) prisitaikyti prie pakitusių 
    sąlygų.
  \item[ektravertiškumas (\gls{ekstraversija}) – intravertiškumas] 
    (Ekstravertams svarbi dabartis, jiems praeitis ir ateitis neegzistuoja 
    – jie linkę atidėti nemalonius darbus.)
\end{description}

Tipų apibūdinimai:

\begin{description}
  \item[Sangvinikas] Stiprus, pusiausvyras, paslankus. 
    \begin{itemize}
      \item Žemas sensityvumas – nepastebi smulkmenų (pavyzdžiui, kito 
        sielvarto).
      \item Padidėjęs reaktyvumas – į viską, kas patraukia jo dėmesį ryškiai
        reaguoja, aiškūs judesiai, gali staigiai perkelti dėmesį.
      \item Aktyvus.
      \item Didelis reakcijos tempas.
      \item Geras plastiškumas – greitai pertvarko įpročius.
    \end{itemize}
  \item[Cholerikas] Stiprus, nėra pusiausvyras.
    \begin{itemize}
      \item Žemas sensityvumas.
      \item Aukštas reaktyvumas (dominuoja) – nesuvaldomas, nekantrus, 
        staigus.
      \item Aukštas aktyvumas.
      \item Didelis psichinės veiklos tempas.
      \item Labiau rigidiškas, nei sangvinikas – sunkiai perkelia dėmesį
        (dažnai keisti veiklos nepatartina).
    \end{itemize}
  \item[Flegmatikas] Stiprus, pusiausvyras, inertiškas.
    \begin{itemize}
      \item Žemas sensityvumas (nejautrus).
      \item Žemas reaktyvumas (ne emocionalus) – sunkiai reaguoja į išorės
        smulkmenas, amžinai rimtas; sunku supykinti, dar sunkiau nustebinti.
        (Geras paskaitų kokybės kriterijus: jei pavyko uždegti flegmatiko
        akis, tai dėstytojas – geras.) Mimika ir judesiai neišraiškūs.
        Kantrus, santūrus, savitvardus, sunkiai įsisavina naują medžiagą,
        bet išmoksta visam gyvenimui.
      \item Aukštesnis aktyvumas už reaktyvumą (?) – neimlus išorės 
        įspūdžiams, svarbus vidinis pasaulis, mąslus, rimtas analitikas.
      \item Rigidiškas – pastovus, pergyvena sprendimo keitimą.
      \item Intravertas – sunkiai bendraujantis, neišraiškus, patikimas.
    \end{itemize}
  \item[Melancholikas] Silpnas.
    \begin{itemize}
      \item Aukštas sensityvumas – jautrus.
      \item Mažas reaktyvumas.
      \item Lėtas.
      \item Rigidiškas.
      \item Intravertiškas – sunkiai bendrauja, į kompaniją prisijungia
        paskutinis.
      \item Neišraiškus, retai juokiasi, linkęs nepasitikėti savimi, 
        neatkaklus, greitai nuvargsta, sunkiai išlaiko dėmesį.
    \end{itemize}
\end{description}

Visi temperamento tipai turi vienodas galimybes lipti karjeros laiptais tik
jie renkasi skirtingas priemones.

\section{Charakteris}

\Gls{charakteris} – \glsentrydesc{charakteris}. (Pavyzdžiui, jei pora
nusprendžia kurti šeimą, tai jiems pasikeičia situacija ir gali išlįsti
tokios charakterio savybės, kurios net nebuvo numanomos.)

Esminis skirtumas tarp temperamento ir charakterio yra tas, kad 
temperamentas yra įgimtas, o charakteris yra įgytas. Charakteris formuojasi
užsiimant veikla.

Charakterio savybės būna skirtingo intensyvumo. Pačios stipriausios
(intensyviausios) charakterio savybės formuojasi konfliktinėse situacijose.
(Pavyzdžiui, nekonfliktinėse situacijose suformuotą „stropios mokinės“ 
charakterį gali greitai perlaužti pasikeitusi aplinka.)

Charakteris yra kintantis dalykas. Jei charakterio savybei yra neleidžiama
pasireikšti (ar asmenine valia, ar išorine), tai ji silpnėja ir išnyksta.
Tuo paremtas auklėjimas.

% FIXME Išsiaiškinti++
Charakterio bruožų skirstymas pagal pagrindines veiklas (charakterio
simptomų kompleksai):

\begin{enumerate}
  \item bendravimas – užuojauta (empatija), jautrumas, reiklumas;
  \item darbas;
  \item požiūris į daiktus;
  \item bendravimas su pačiu savimi – genialus, išdidus, kuklus, savimi
    (ne)pasitikintis; jei dėstomą medžiagą nepaaiškina vidinis balsas –
    nesupranti.
\end{enumerate}

% FIXME Išsiaiškinti.
Charakterio savybės skiriasi:

\begin{enumerate}
  \item amplitude;
  \item plastiškumu (kaip lengvai keičiasi);
  \item raida.
\end{enumerate}

Hipotezė: vaikui reikalingos kitos charakterio savybės, nei suaugusiajam,
todėl charakterio bruožai įgyti iki penkerių metų yra išstumiami.

\section{(Su)gebėjimai}

\Gls{gebejimai} – \glsentrydesc{gebejimai}.

Sugebėjimų komponentai:

\begin{enumerate}
  \item įgytieji;
  \item įgimtieji.
\end{enumerate}

Gebėjimų kompensacijos fenomenas – jei žmogus yra gabus vienoje srityje, 
tai jis „turi problemų“ kitose.

\section{Asmenybė}

\label{tema:asmenybe}

\emph{Žmogų} paprastai suprantame kaip biologinę būtybę, kuri skiriasi nuo
gyvulių tuo, kad turi sąmonę, sugeba pažinti pasaulį bei save, sugeba 
protingai veikti. Priklausomybė žmonių giminei psichologijoje yra
nusakoma \emph{individo} sąvoka.

\Gls{asmenybe} tai \glsentrydesc{asmenybe}. Ją taip pat galima apibrėžti 
kaip psichinių savybių sistema, kurią sudaro: \emph{kryptingumas}, 
\emph{sugebėjimai}, \emph{temperamentas} bei \emph{charakteris}.
(\emph{Asmenybės kryptingumas} – apima visa tai, ko žmogus siekia, ir to 
siekimo priežastis.)

Aktyvumo problema: „kas yra žmogaus aktyvumo šaltinis, kokios jo 
priežastys?“. Šiuo klausimu diskutuojama jau senai, bet XX a. diskusijos 
suaktyvėjo. Z. Froido nuomone, žmogaus aktyvumą lemia iš gyvulių paveldėti 
lytinis ir savisaugos instinktai. Žmonių visuomenėje jie nebegali taip 
lengvai reikštis dėl daugybės visuomeninių apribojimų, todėl jie išstumiami 
iš sąmoningo elgesio kaip neleistini. Tačiau jie pereina į pasąmonę ir 
toliau valdo žmogaus elgesį. Neofroidistai neigia atvirą žmogaus veiklos 
biologizavimą ir teigia, kad aktyvumo šaltinis yra poreikiai.
Taigi aktyvumas būna dėl:
\begin{itemize}
  \item motyvacijos veiklai (esam atsinešę iš patirties);
  \item veiklos tikslo turėjimo.
\end{itemize}
Motyvacija stumia, tikslas traukia, todėl žmogaus veikla yra tikslinga
(kryptinga).

Motyvams priklauso:
\begin{itemize}
  \item poreikiai (valgiui, saugumui, šilumai) – anksčiausiai 
    pasireiškiantys motyvai. Poreikius tenkina tam tikri objektai skirti
    mūsų manipuliacijoms, todėl objektai gali turėti realiąją (ekonominę,
    kultūrinę – paveldas, infrastruktūrinę – turizmas) ir poreikinę 
    (stabiliąją bei dinaminę) reikšmes;
    % FIXME stabilioji – pavyzdžiui, žmogus gali turėti stabilųjį poreikį 
    % mokslinei literatūrai.
    % FIXME dinaminė – poreikį tenkinant, jis mažėja, bet visai neišnyksta,
    % o po kiek laiko atgimsta subtilesne forma.

    Poreikiai gali būti klasifikuojami į:
    \begin{itemize}
      \item neįsisąmonintus – nuostatos, potraukiai;
      \item įsisąmonintus – idealai, sektini pavyzdžiai, įsitikinimai,
        interesai (intereso pavyzdys – emocinis pažinimo motyvas);
    \end{itemize}

    Poreikiai taip pat gali būti stabilūs ir nestabilūs.
\end{itemize}

% FIXME Išsiaiškinti prie ko šitai:
% Sisteminė asmenybės savybė – konformizmas, įtaigumas. 
% konformizmas – grupinis įtaigumas, sutikimas su grupės nuomone. (Gėris
% ir blogis yra konformizmo nuostatos.)

\Gls{aspiraciju_lygis} – \glsentrydesc{aspiraciju_lygis}. Jei žemas 
aspiracijų lygis, tai žmogus yra nepasitikintis savo jėgomis, nerealizuoja
savęs. Jei aukštas, tai žmogus save pervertina, kelia sau per aukštus
tikslus, yra garbėtroška, pagyrūnas.

Asmenybės raidos teorijų grupės:
\begin{description}
  \item[biogenetinės] – akcentuoja paveldimumą (nebūtinai genus), turi
    daug empirinių pagrindimų;
  \item[socialgenetinės] – akcentuoja patyrimą (teigia, kad žmogus gimsta
    kaip tuščia lenta);
  \item[konvergencinės teorijos] – biogenetinių ir socialgenetinių mišinys;
  \item[formavimosi teorijos]  – žmogus formuojasi siekdamas tikslų, darbu.
    Kuo sunkiau asmenybei pasiekti tikslą, juo stipresnės savybės 
    formuojasi.
\end{description}

Asmenybės veiklos rūšys:

\begin{description}
  \item[impulsyvi] – nėra aiškaus motyvo, bet yra neįsisąmoninta paskata;
  \item[praktinė] – darbo pobūdžio, duoda prielaidas poreikių tenkinimui,
    kuria išliekamąją vertę; % FIXME Kieno išliekamąją vertę?
  \item[mokymosi ir treniravimosi] – formuoja žinias, įgūdžius;
  \item[žaidimo] – fizinė veikla (kai kurie psichologai nelinkę išskirti,
    kaip atskiros veiklos rūšies, bet žaidimai turi smegenyse atskiras 
    struktūras ir netgi juos galima paveikti vaistais – pavyzdžiui, 
    priversti mirtinai užsižaisti).
  % FIXME Ar yra ir šitas: \item[kūryba] ?
\end{description}

% TODO Išsiaiškinti:
Interiorizuota veikla (vidinė).
Eksteriorizuota veikla (išorinė).
Visi žmonių sukurti daiktai yra interiorizacijos ir eksteriorizacijos
pasekoje.

Įgūdžiai turi tris komponentus:

\begin{description}
  \item[vykdymo] Formuojantis įgūdžiams, jungiasi keli atskiri fiziniai
    judesiai į vientisą bendrą judesį. (Dingsta judesių „kampuotumas“, 
    atsiranda „glotnumas“, „tolydumas“ – itin gerai matosi šokiuose.)
  \item[kontrolės] 
    % FIXME Kas būtent yra kontrolė ir kuo ji skiriasi nuo valdymo.
    Iš pradžių atliekant veiklą (mokantis vairuoti automobilį)
    reikia sąmoningai ieškoti (stabdžių pedalo, pavarų svirties), o vėliau
    tie dalykai susirandami automatiškai.
  \item[valdymo] Dėmesys keičiasi į rezultatą, nebereikia visą laiką
    būti sutelkus dėmesio į vykdymą – valdymas iš taktikos pereina į
    strategiją.
\end{description}

Turimi įgūdžiai su naujai įgyjamais gali sąveikauti dviem būdais:

\begin{itemize}
  \item indukcijos – seni įgūdžiai padeda naujiems (mokėjimas važiuoti
    dviračiu praverčia mokantis važiuoti motociklu);
  \item interferencijos – seni įgūdžiai trukdo naujiems (rašymas 
    veidrodiškai).
\end{itemize}

% FIXME Mokėjimas – žinių ir įgūdžių pasirinkimas?

Išmokimui (tiek įgūdžiams, tiek žinioms įsisavinti) yra reikalingas 
kartojimas. Bet ne pats formalus kartojimas, o tai, kad jis leidžia
pasireikšti praktiniam mąstymui. Yra tokių išmokimų, kuriems kartojimas
nepadeda.
