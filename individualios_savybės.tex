\chapter{Individualios savybės}

\section{Temperamentas}

Tiek žmonės, tiek gyvūnai vieni nuo kitų skiriasi pagal savo psichinės 
veiklos greitį. Šio greičio aprašymui naudojamas \gls{temperamentas}. 
Bet greitis priklauso ir nuo individo motyvacijos bei psichinės būsenos. 
Esminis skirtumas yra tai, kad temperamentas yra pastovus.

Temperamentą nulemia % FIXME Ar tikrai „nulemia“?
sumariniai (integraciniai) galvos smegenų procesai:

\begin{itemize}
  % TODO Papildyti.
  \item jaudinimas; 
  \item slopinimas.
\end{itemize}

Integracinių procesų parametrai:

\begin{itemize}
  \item amplitudė – skirtumas tarp mažiausio ir didžiausio smegenų dalies
    susijaudinimų;
  \item pusiausvyrumo laipsnis:
    \begin{itemize}
      \item pusiausvyras – visą laiką yra sujaudinta maždaug vienodo
        dydžio smegenų dalis;
      \item nepusiausvyras – galima išskirti ne tik dalių aktyvumo pokyčius,
        bet ir pačių smegenų.
    \end{itemize}
  \item paslankumas – FIXME
\end{itemize}

% FIXME Variantai, šiaip tai aštuoni – kiekvienas gali įgyti po dvi
% reikšmes, iš viso trys, tai 2^3 = 8?
Teoriškai pagal tris parametrus galima būtų išskirti šešis tipus, bet
jei nervų sistema yra silpna, tai neišeina normaliai nustatyti kitų dviejų
parametrų, todėl yra išskiriami tik keturi tipai. Jie iki šiol įvardijami
dar pagal Hipokrato terminologiją:

\begin{itemize}
  \item sangvinikas – stipri, pusiausvyra ir judri nervų sistema;
  \item cholerikas – stipri, nepusiausvyra (stiprus jaudinimas ir 
    silpnas slopinimas) nervų sistema;
  \item flegmatikas – stipri, pusiausvyra, lėta nervų sistema;
  \item melancholikas – silpna nervų sistema.
\end{itemize}

Kadangi gyvenimas priverčia žmogų į situacijas reaguoti tinkamai 
(aplinkos, kuriai priklauso žmogus požiūriu), tai tikruosius temperamento
tipus galima būtų matyti tik pas iki mokyklinio amžiaus vaikus.

Norint nustatyti temperamento tipą yra matuojami (Plačiau –
\url{http://lt.wikipedia.org/wiki/Temperamentas}):

\begin{description}
  \item[sensityvumas (jautrumas)] Matuojamas apatinis pojūčio slenkstis;
    minimalus poreikio nepatenkinimo laipsnis, sukeliantis vos juntamą
    kentėjimą.
  \item[reaktyvumas (emocionalumas)] Matuojama kokio stiprumo emocijas
    sukelia žmogui išorinis arba vidinis analogiško stiprumo poveikis.
  \item[aktyvumas] Jį apibūdina tai, kaip žmogus veikia, nugalėdamas 
    išorines ir vidines kliūtis.
  \item[aktyvumo ir reaktyvumo santykis] Jį parodo vyraujantis veiklos
    akstinas – nuo ko daugiau priklauso žmogaus veikla: nuo atsitiktinių
    išorinių aplinkybių (pirmauja reaktyvumas) ar nuo vidinių (pirmauja
    aktyvumas).
  \item[reakcijos tempas (ne greitis!)] Jį parodo reakcijos laikas į 
    šviesos, garso stimulus, į prasmės suvokimą.
  \item[plastiškumas – rigidiškumas] Jį parodo kaip greitai žmogus sugeba
    (plastiškumas) ar nesugeba (rigidiškumas) prisitaikyti prie pakitusių 
    sąlygų.
  \item[ektravertiškumas (\gls{ekstraversija}) – intravertiškumas] 
    (Ekstravertams svarbi dabartis, jiems praeitis ir ateitis neegzistuoja 
    – jis linkęs atidėti nemalonius darbus.)
\end{description}

Tipų apibūdinimai:

\begin{description}
  \item[Sangvinikas] Stiprus, pusiausvyras, paslankus. 
    \begin{itemize}
      \item Žemas sensityvumas – nepastebi smulkmenų (pavyzdžiui, kito 
        sielvarto).
      \item Padidėjęs reaktyvumas – į viską, kas patraukia jo dėmesį ryškiai
        reaguoja, aiškūs judesiai, gali staigiai perkelti dėmesį.
      \item Greitas.
      \item Geras plastiškumas – greitai pertvarko įpročius.
    \end{itemize}
  \item[Cholerikas] Stiprus, nėra pusiausvyras.
    \begin{itemize}
      \item Žemas sensityvumas.
      \item Aukštas reaktyvumas (dominuoja).
      \item Aukštas aktyvumas,
      \item Greitas.
      \item Labiau rigidiškas, nei sangvinikas – sunkiai perkelia dėmesį
        (dažnai keisti veiklos nepatartina).
    \end{itemize}
  \item[Flegmatikas] Stiprus, pusiausvyras, inertiškas.
    \begin{itemize}
      \item Žemas sensityvumas (nejautrus).
      \item Žemas reaktyvumas (ne emocionalus) – sunkiai reaguoja į išorės
        smulkmenas, amžinai rimtas; sunku supykinti, dar sunkiau nustebinti.
        (Geras paskaitų kokybės kriterijus: jei pavyko uždegti flegmatiko
        akis, tai dėstytojas – geras.) Kantrus, santūrus, savitvardus,
        sunkiai įsisavina naują medžiagą, bet išmoksta visam gyvenimui.
      \item Aukštesnis aktyvumas už reaktyvumą (?) – neimlus išorės 
        įspūdžiams, svarbus vidinis pasaulis, mąslus, rimtas analitikas.
      \item Rigidiškas – pastovus, pergyvena sprendimo keitimą.
      \item Intravertas – sunkiai bendraujantis, neišraiškus, patikimas.
    \end{itemize}
  \item[Melancholikas] Silpnas.
    \begin{itemize}
      \item Aukštas sensityvumas – jautrus.
      \item Mažas reaktyvumas.
      \item Lėtas.
      \item Rigidiškas.
      \item Intravertiškas – sunkiai bendrauja, į kompaniją prisijungia
        paskutinis.
      \item Neišraiškus, retai juokiasi, linkęs nepasitikėti savimi, 
        neatkaklus, greitai nuvargsta, sunkiai išlaiko dėmesį.
    \end{itemize}
\end{description}

Visi temperamento tipai turi vienodas galimybes lipti karjeros laiptais tik
jie renkasi skirtingas priemones.

\section{Charakteris}

Temperamentas yra įgimtas. Charakteris yra įgytas.

\section{Gebėjimai}

