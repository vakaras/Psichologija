\chapter{Įvadas}

\section{Psichologijos vieta mokslų sistemoje}

\label{tema:psichologijos_mokslas}

Žodis „psichologija“ sudarytas iš dviejų graikų kalbos žodžių: 
\emph{psyche} (gyvybė, siela) ir \emph{logos} (mokslas). Taigi, verčiant
pažodžiui, psichologija yra mokslas apie sielą.

Galime išskirti buitinę ir mokslinę psichologiją. Mokslinei psichologijai
būdinga:
\begin{itemize}
  \item metodiškumas – žinios kaupiamos sąmoningai, tikslingai atliekant
    tyrimus ir bandymus;
  \item sistemingumas – bandoma sukurti sistemą paaiškinančią visus 
    stebimus reiškinius (žinios yra apibendrinamos, siekiama rasti 
    universalius principus, priežasties – pasekmės ryšius ir t.t.)
  \item tikslumas – visos naudojamos sąvokos turi konkrečią, tiksliai
    apibrėžtą prasmę.
\end{itemize}
Buitinei psichologijai būdinga:
\begin{itemize}
  \item padrikumas – žinios kaupiamos iš atsitiktinių pastebėjimų, 
    stebint save ir kitus;
  \item žinios konkrečios, pritaikytos konkrečioms atskiroms situacijoms;
  \item vartojamos sąvokos turi „išplaukusią“, be konkrečių ribų 
    semantinę reikšmę.
\end{itemize}

\section{Psichologijos tyrimų objektas}

\label{tema:psichikos_samprata}

TODO: Psichika – apibrėžti.
TODO: Psichikos multidimensiškumas – apibrėžti.

\Gls{psichologija} nagrinėja tris pagrindines psichinių reiškinių grupes:

\begin{enumerate}
  \item psichiniai procesai (\ref{tema:psichiniai_procesai} skyrius):
    \begin{itemize}
      \item \glspl{pojutis},
      \item suvokimas (ang. perception),
      \item atmintis,
      \item mąstymas,
      \item vaizduotė;
    \end{itemize}
  \item psichinės būsenos (\ref{tema:psichines_busenos} skyrius):
    \begin{itemize}
      \item dėmesys,
      \item emocinė aktyvacija;
    \end{itemize}
  \item TODO: individualios asmenybės savybės:
    \begin{itemize}
      \item psichinių procesų ir būsenų individualios variacijos,
      \item „grynai“ individualios savybės:
        \begin{enumerate}
          \item temperamentas,
          \item gebėjimas,
          \item charakteris.
        \end{enumerate}
    \end{itemize}
\end{enumerate}
