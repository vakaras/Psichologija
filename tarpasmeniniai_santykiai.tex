\chapter{Tarpasmeniniai santykiai}

Bendravimo raidos grupės:

\begin{description}
  \item[sąlyginės] – labai kintančios, kontakto dažniausiai nebūna 
    (pavyzdžiui, vienu troleibusu važiuojantys ar viename mieste gyvenantys 
    žmonės);
  \item[realiosios] – asmenų bendroje erdvėje ir laike, pavyzdžiui:
    \begin{itemize}
      \item pati didžiausia reali grupė – žmonija;
      \item istoriškai seniausia darbo grupė – šeima.
    \end{itemize}

    Reali grupė gali turėti variacijas:
    \begin{itemize}
      \item referentinė grupė – suderinama grupė, arba kitaip –
        „reikšmingieji kiti“, su kuriais esame linkę lyginti save 
        (pavyzdžiui, tą pačią paskaitą lankančių studentų grupė);
      \item oficiali grupė – grupė, kurios lyderį paskelbia išoriniai
        veiksniai;
      \item neoficiali grupė – grupė, natūraliai atsirandanti oficialioje
        ar kitose grupėse. Neoficiali grupė išsirenka savo lyderį. Jei
        neoficialios grupės lyderio tikslai ir nuostatos sutampa su
        oficialios grupės, tai tokia grupė labai daug pasiekia, o jei
        nesutampa, tada prasideda kova ir tai mažina grupės darbo 
        efektyvumą.
    \end{itemize}

\end{description}

% FIXME Kalbėti klausa išmokstame anksčiau, nei kalbėti rega.

FIXME

Grupinė ir individuali sąmonė.

Grupinės sąmonės pasireiškimas: bendra idėja, ideologija,
% FIXME Kaip suprasti „grupinės sąmonės labai svarbi forma“
panika, lenktyniavimo fenomenas, visuomenės gerovės jausmas.

Verbalinė veikla arba kalbėjimas.

Patirties rūšys:
\begin{enumerate}
  \item žmonijos sukaupta – verbalinė veikla, kaip žmonijos egzistavimo 
    palaikymo sistema; visos žinios, patyrimas (matematinės formulės
    ir panašiai);
  \item atskirų asmenybių – bendravimas su kitais žmonėmis turint tikslą
    ir panašiai;
  \item asmeninė patirtis – bendravimas su pačiu savimi (vidinis balsas).
\end{enumerate}

\Glspl{afazija} – verbalinės veiklos sutrikimai. Variantai:
\begin{description}
  \item[dinaminė afazija] – negali kalbėti frazėmis;
  \item[eferentinė afazija] – negali sujungti garsų, pažeidžiamas nuoseklumo
    principas;
  \item[aferentinė afazija] – negali rasti tinkamo garso, naudoja tik
    vidutinius…; %FIXME Ką vidutinius?
  \item[semantinė afazija]  – prasmės elekstrakcija, 
    % FIXME Kas yra ta elekstrakcija?
    kai žmogus atskirus žodžius supranta, o sujungto sakinio ne;
  \item[sensorinė afazija] – pakenkiama žodžių garsinė analizė. (Pavyzdžiui,
    jei žmogus apkursta, po kiek laiko nebesuprastume, ką jis kalba, nors
    jam pačiam atrodo, jog jis kalba aiškiai.)
\end{description}
