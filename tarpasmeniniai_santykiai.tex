\chapter{Tarpasmeniniai santykiai}

\label{tema:grupe}

Bendravimo raidos grupės:

\begin{description}
  \item[sąlyginės] – labai kintančios, kontakto dažniausiai nebūna 
    (pavyzdžiui, vienu troleibusu važiuojantys ar viename mieste gyvenantys 
    žmonės);
  \item[realiosios] – asmenų bendroje erdvėje ir laike, pavyzdžiui:
    \begin{itemize}
      \item pati didžiausia reali grupė – žmonija;
      \item istoriškai seniausia darbo grupė – šeima.
    \end{itemize}

    Reali grupė gali turėti variacijas:
    \begin{itemize}
      \item referentinė grupė – suderinama grupė, arba kitaip –
        „reikšmingieji kiti“, su kuriais esame linkę lyginti save 
        (pavyzdžiui, tą pačią paskaitą lankančių studentų grupė);
      \item oficiali grupė – grupė, kurios lyderį paskelbia išoriniai
        veiksniai;
      \item neoficiali grupė – grupė, natūraliai atsirandanti oficialioje
        ar kitose grupėse. Neoficiali grupė išsirenka savo lyderį. Jei
        neoficialios grupės lyderio tikslai ir nuostatos sutampa su
        oficialios grupės, tai tokia grupė labai daug pasiekia, o jei
        nesutampa, tada prasideda kova ir tai mažina grupės darbo 
        efektyvumą.
    \end{itemize}

\end{description}

% FIXME Kalbėti klausa išmokstame anksčiau, nei kalbėti rega.

FIXME

Grupinė ir individuali sąmonė.

Grupinės sąmonės pasireiškimas: bendra idėja, ideologija,
% FIXME Kaip suprasti „grupinės sąmonės labai svarbi forma“
panika, lenktyniavimo fenomenas, visuomenės gerovės jausmas.

\section{Kalba ir kalbėjimas. Jų funkcijos ir rūšys}

Verbalinė veikla arba kalbėjimas.

Patirties rūšys:
\begin{enumerate}
  \item žmonijos sukaupta – verbalinė veikla, kaip žmonijos egzistavimo 
    palaikymo sistema; visos žinios, patyrimas (matematinės formulės
    ir panašiai);
  \item atskirų asmenybių – bendravimas su kitais žmonėmis turint tikslą
    ir panašiai;
  \item asmeninė patirtis – bendravimas su pačiu savimi (vidinis balsas).
\end{enumerate}

\Glspl{afazija} – verbalinės veiklos sutrikimai. Variantai:
\begin{description}
  \item[dinaminė afazija] – negali kalbėti frazėmis;
  \item[eferentinė afazija] – negali sujungti garsų, pažeidžiamas nuoseklumo
    principas;
  \item[aferentinė afazija] – negali rasti tinkamo garso, naudoja tik
    vidutinius…; %FIXME Ką vidutinius?
  \item[semantinė afazija]  – prasmės elekstrakcija, 
    % FIXME Kas yra ta elekstrakcija?
    kai žmogus atskirus žodžius supranta, o sujungto sakinio ne;
  \item[sensorinė afazija] – pakenkiama žodžių garsinė analizė. (Pavyzdžiui,
    jei žmogus apkursta, po kiek laiko nebesuprastume, ką jis kalba, nors
    jam pačiam atrodo, jog jis kalba aiškiai.)
\end{description}


\label{tema:kalba_ir_kalbejimas}

\subsection{Kalba ir kalbėjimas}

Žmonės savo psichikos turinį (mintis, vaizdinius, emocijas ir kt.)
kitiems žmonėms perduoda, pasinaudodami įvairiais materialiais objektais
(garsais, spalvomis, judesiais, daiktais). Tie materialūs objektai, kurie
žymi tam tikrą psichikos turinį, yra vadinami ženklais. \emph{Kalba} yra
svarbiausia bendravimui skirta ženklų sistema. Ženklų funkcijas kalboje
atlieka ne tik žodžiai, bet ir tų žodžių jungimo taisyklės. Kalbantysis 
gali perduoti savo santykį su reiškiamomis mintimis per kalbos intonaciją.
Garsinėje kalboje ji reiškiama įvairiais balso pakitimais, regimoje kalboje
– skyrybos, kirčių, skirtingų šriftų ir kt. ženklais.

\emph{Kalbėjimas} yra žmonių bendravimas kalbos ženklais. Šiame bendravime
dalyvauja dvi pusės: kalbantysis (komunikatorius) ir kalbėjimą 
suprantantysis (percipientas). Kad bendraujantys žmonės galėtų kalbėtis, 
jie turi mokėti tą pačią kalbą. Bendraujančiųjų atmintyje turi būti 
„įrašytas“ tam tikras kiekis tos sistemos ženklų, kurie prireikus
turi būti greitai „išvedami“. Todėl kalbėjimuisi pirmiausia reikia 
automatizuotų veiksmų – kalbėjimo įgūdžių.

Kalbėjimas (bendravimas) atlieka tokias funkcijas:
\begin{itemize}
  \item komunikacinę – keitimasis įvairiausia informacija (mintimis,
    vaizdiniais, emocijomis, interesais, įgūdžiais, nuostatomis ir
    pan.)
  \item interakcinę – asmenų tarpusavio sąveika (gali padidėti arba 
    sumažėti bendraujančiųjų aktyvumas, atsirasti teigiamų ir neigiamų
    emocijų);
  \item percepcinę – savitarpio suvokimas ir geresnis pažinimas.
\end{itemize}

\subsection{Kalbėjimo rūšys}

\begin{itemize}
  \item \emph{Vidiniu kalbėjimu}, arba kalbėjimu sau, pasirengiama 
    išoriniam kalbėjimui, iš anksto planuojami būsimi pasisakymai. Be to, 
    juo remiamasi, sprendžiant žodinius uždavinius, kai valios ir mąstymo 
    procesuose kyla kokių nors sunkumų.
  \item \emph{Išorinis kalbėjimas} skirtas bendrauti su kitais asmenimis,
    todėl turi būti pakankamai intensyvus, išplėstas, gausus įvairių 
    intonacijų ar regimų ženklų, kad percipientai aiškiau suprastų.  
    Pagrindinės išorinio kalbėjimo formos yra sakytinis ir rašytinis 
    kalbėjimas.
    \begin{itemize}
      \item \emph{Sakytinį kalbėjimą} sudaro girdimų signalų grupė. Kadangi 
        garsų ypatybės gali kisti labai plačiu diapazonu, susidaro 
        kalbėjimo garsinių signalų įvairovė. Tai įgalina reikšti ne tik 
        mintis, bet ir jausmus bei nuostatą kalbamojo dalyko atžvilgiu. 
        Sakytinis kalbėjimas skirstomas į \emph{dialoginį} ir 
        \emph{monologinį}.
        \begin{itemize}
          \item \emph{Dialoginis kalbėjimas} dar vadinamas situaciniu arba 
            kontekstiniu. Jis yra nesudėtinga kalbinio bendravimo forma, 
            nes jam nebūtina iš anksto planuoti pasakymus, jis neturi 
            vieningo tikslo.
          \item \emph{Monologiniame kalbėjime} ilgesnį laiką kalba vienas 
            asmuo, o kalbėjimą suvokiantieji patys aktyviai jame 
            nedalyvauja.  Kalbėtojas turi numatyti kalbėjimo tikslą ir 
            programą. 
        \end{itemize}
      \item \emph{Rašytinis kalbėjimas} atsirado iš sakytinio, siekiant 
        išlaikyti mintis ilgesnį laiką. Rašymo ir skaitymo nevaržo tempas, 
        todėl rašantysis gali geriau parinkti žodžius, sakinio 
        konstrukcijas, atlikti redakcinius taisymus, taip pasiekdamas 
        geriausią minčių išraiškos variantą. Tuo tarp skaitytojas gali 
        paskaityti pakartotinai, apžvelgti tolimesnį kontekstą. Tačiau 
        rašytinis kalbėjimas neturi tokių turtingų išraiškos priemonių, 
        kaip sakytinis.
    \end{itemize}
\end{itemize}
\section{Konfliktai}

\label{tema:konfliktai}

\emph{Konfliktas} – tai priešingų pažiūrų, interesų, emocijų susidūrimas, 
atsiradęs žmonėms sprendžiant socialinio ar asmeninio pobūdžio klausimus.
Konfliktas ne visada yra neigiamas reiškinys. Kai nuomonių susidūrimas 
liečia svarbius veiklos tikslus, jis įgalina rasti prasmingą sprendimą. 
Toks produktyvus konfliktas stiprina socialinį aktyvumą ir padeda 
išsiveržti iš rutinos. Neigiamais konfliktais laikomi nesutarimai tarp 
asmens ir grupės, vadovo ir pavaldinio. Tokiais atvejais konfliktas 
suaktyvina prieštaraujančių pusių nesutarimus bei išardo visuomeninę darną. 
Konfliktai gali būti trumpalaikiai arba ilgalaikiai. Kartais dviejų žmonių 
konfliktas deorganizuoja grupės veiklą, neigiamai veikia žmonių nervinę 
būseną.

Konfliktai gali būti sprendžiami dviem būdais:
\begin{itemize}
  \item \emph{pedagoginio} konfliktų sprendimo būdo pagrindas yra 
    įtikinimas. Klystančiai arba abiems konfliktuojančioms pusėms padedama 
    suvokti objektyvią tiesą;
  \item \emph{administraciniu} konflikto sprendimo būdu yra grupės nariui 
    keliamas ultimatumas keisti savo elgesį arba asmens perkėlimas į kitą 
    kolektyvą.
\end{itemize}
