\chapter{Psichinės būsenos}

\label{tema:psichines_busenos}

\section{Emocijos, jausmai, valia}

Egzistuoja dvi pasaulio vertinimo sistemos: emocinė (senesnė) ir loginė 
(naujesnė). Gyvūnai turi tik emocijas, o žmonės – ir emocijas, ir jausmus. 
\Gls{emocija} \glsentrydesc{emocija}. (Emocija plačiąja prasme apima
ir emocijas, ir jausmus.) \Gls{jausmas} yra socializuota 
\gls{emocija}. Plačiau, \gls{jausmas} tai yra \glsentrydesc{jausmas}.

\label{tema:emocijos}
Atsižvelgiant į emocinių reiškinių lygį pagal jų kilmę, jie skirstomi į:
\begin{description}
  \item[žemesniuosius jausmus] – susiję su biologiniais organizmo 
    poreikiais (dar vadinami elementariosiomis emocijomis);
  \item[aukštesniuosius jausmus] – susiję su visuomeniniais veiksniais.
    Jie dar skirstomi į rūšis (doroviniai, intelektiniai, estetiniai,
    praktiniai) pagal juos sukėlusių objektų ypatumus.
\end{description}

Pagal emocinių išgyvenimų trukmę, stiprumą ir kilmę skiriamos tokios
jų formos (išraiškos):
\begin{description}
  \item[\gls{nuotaika}] – \glsentrydesc{nuotaika} (ilgalaikė);
  \item[\gls{afektas}] – \glsentrydesc{afektas} (trumpalaikė);
  \item[\gls{aistra}] – \glsentrydesc{aistra} (trumpalaikė);
  \item[\gls{stresas}] – \glsentrydesc{stresas} (trumpalaikė).
\end{description}

Emocijos atlieka dvi pagrindines funkcijas:
\begin{description}
  \item[impresyvioji] (signalinė) – objekto atžvilgiu kilusi emocija
    žmogui signalizuoja apie tai, kaip tas objektas tenkina poreikius;
  \item[ekspresyvioji] (reguliuojamoji) – emocijų išraiškomis (veido
    mimika, kalbos intonacija ir t.t.) žmogus informuoja kitus individus 
    apie savo būseną, skatina poreikių tenkinimą.
\end{description}

Jausmų savybės:
\begin{itemize}
  \item poliariškumas – gali būti teigiami ir neigiami, aktyvūs ir pasyvus
    (neigiamos emocijos gali skatinti veikti (pavyzdžiui, pulti), bet taip 
    pat gali ir slopinti);
  \item dvilypumas – tuo pačiu metu to paties objekto atžvilgiu gali būti
    juntami priešingi jausmai (pavyzdžiui, stiprus pavydas sukelia 
    vieningą meilės ir neapykantos jausmą);
  \item orientacinis neapibrėžtumas – emocijos užplūsta tada, kai būna
    neaišku. Jei viskas pasidaro aišku, emocijų nelieka.
\end{itemize}

\section{Dėmesys}

\label{tema:demesys}

Plačiau: \url{http://lt.wikipedia.org/wiki/Dėmesys}

\Gls{demesys} – \glsentrydesc{demesys}.

Dėmesio rūšys (pagal valios laipsnį):

\begin{enumerate}
  \item nevalingas – mažai varginantis, visiškai priklausantis nuo stimulo, 
    šokinėjantis;
  \item valingas – labai varginantis, kryptingas, reikalingas tada, kai 
    turime užsiimti mūsų nedominančia veikla;
  \item savaiminis – dėmesys, kuris atsiranda tada, kai ilgiau padirbėjus
    atsiranda susidomėjimas iš pradžių buvusia neįdomia veikla.
\end{enumerate}

Kartais perėjimas nuo nevalingo prie savaiminio dėmesio dar yra vadinamas
dėmesio raida.

Dėmesio atliekamos funkcijos:

\begin{itemize}
  \item atrenka iš sudėtingos aplinkos tuos poveikius, kurie dėl tam tikrų
    priežasčių yra reikšmingesni už kitus;
  \item koncentruoja ir palaiko psichines jėgas ties pasirinktais objektais;
  \item reguliuoja išankstinį pasirengimą veiklai, jos vyksmą ir 
    pertraukimą.
\end{itemize}

%Orientacinis refleksas skirtas iš visos informacijos srauto išskirti tai,
%kas yra nauja.

%Valingas dėmesys gali be pertraukimų išsilaikyti ne daugiau, kaip 5 minutes.

%Dėmesio savybės (dėstytojo):

%\begin{itemize}
  %\item koncentracija – atvirkštinė susikaupimo zonos dydžiui;
  %\item visaapimiškumas – ar sistemą suvokiam, kaip dalių krūvą, ar kaip
    %visumą; % TODO Pavyzdys apie mašinos vairavimą.
  %\item pastovumas –;
  %\item automatizmas –.  % TODO Pavyzdys apie savaime grojančias rankas.
%\end{itemize}

\label{tema:demesio_savybes}

\Glspl{demesio_savybe} (konspektų):

\begin{description}
  \item[intensyvumas / koncentracija] dėmesio sutelkimas į objektą, 
    atsiribojimas nuo pašalinių poveikių;
  \item[patvarumas] nenutrūkstamas dėmesio sutelkimas ilgesnį laiką 
    į vieną objektą (dėmesys svyruoja kas 1-2 sek., bet atliekant kokią
    nors prasmingą veiklą, dėmesys gali išbūti patvarus net iki 15-20 
    minučių);
  \item[apimtis] kiek objektų dėmesys gali apimti vienu metu (tyrimais
    nustatyta, kad žmogaus dėmesys vienu metu gali apimti $7 \pm 2$ 
    objektus);
  \item[paskirstymas] žmogus vienu metu gali atlikti skirtingus veiksmus,
    kiekvienam veiksmui skirdamas dalį dėmesio (pavyzdžiui, to reikia 
    grojant pianinu, vairuojant automobilį ir pan.);
  \item[perkėlimas] dėmesio nukreipimas iš vieno objekto į kitą
    (gali būti valingas ir nevalingas);
\end{description}

% Naudingas pastebėjimas.
%Darbus derėtų daryti jų įdomumo didėjimo tvarka.

%Dėmesys yra tampriai susietas su emocijomis.

Baimės fazės:

\begin{enumerate}
  \item asteninė (\gls{astenija}) sustingimo reakcija (suakmenėjimas, 
    drąsus šią fazę pereina greičiau, bailus – lėčiau);
  \item panika;
  \item afektas (trumpalaikis emocinis sprogimas);
  \item stresas (emocinė įtampa);
  \item \gls{frustracija}, gali būti viena iš formų:
    \begin{itemize}
      \item \gls{regresija} (verkimas);
      \item agresija (sukeliančios priežasties sunaikinimas)
      \item tendencija automatizmui;
      \item tolerancija – neišsprendžia problemos, o tik ją atideda.
    \end{itemize}
\end{enumerate}
