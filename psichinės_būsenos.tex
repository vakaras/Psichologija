\chapter{Psichinės būsenos}

\label{psichines_busenos}

\section{Emocijos, jausmai, valia}

Egzistuoja dvi pasaulio vertinimo sistemos: emocinė (senesnė) ir loginė 
(naujesnė). 

Gyvūnai turi tik emocijas, o žmones ir emocijas ir jausmus. Jausmas yra
socializuota emocija. Emocijos atlieka dvi pagrindines funkcijas:

\begin{itemize}
  \item impresyvioji – mes perimam;
  \item ekspresyvioji – mes išreiškiam.
\end{itemize}

Jausmų savybės:

\begin{itemize}
  \item dvilypumas – % TODO
  \item orientacinis neapibrėžtumas – emocijos užplūsta tada, kai būna
    neaišku. Jei viskas pasidaro aišku, emocijų nelieka.
\end{itemize}

\section{Dėmesys}

\Gls{demesys} – \glsentrydesc{demesys}.

Dėmesio rūšys (pagal valios laipsnį):

\begin{itemize}
  \item nevalingas – mažai varginantis, visiškai priklausantis nuo stimulo, 
    šokinėjantis;
  \item valingas – labai varginantis, kryptingas;
  \item savaiminis, povalinis.  % TODO Išsiaiškinti.
\end{itemize}

Orientacinis refleksas skirtas iš visos informacijos srauto išskirti tai,
kas yra nauja.

Valingas dėmesys gali be pertraukimų išsilaikyti ne daugiau, kaip 5 minutes.

Dėmesio savybės:

\begin{itemize}
  \item koncentracija – atvirkštinė susikaupimo zonos dydžiui;
  \item vasapiškumas – ar sistemą suvokiam, kaip dalių krūvą, ar kaip
    visumą; % TODO Pavyzdys apie mašinos vairavimą.
  \item pastovumas –;
  \item automatizmas –.  % TODO Pavyzdys apie savaime grojančias rankas.
\end{itemize}

% Naudingas pastebėjimas.
Darbus derėtų daryti jų įdomumo didėjimo tvarka.

Dėmesys yra tampriai susietas su emocijomis.

Baimės fazės:

\begin{enumerate}
  \item asteninė (\gls{astenija}) sustingimo reakcija (suakmenėjimas, 
    drąsus šią fazę pereina greičiau, bailus – lėčiau);
  \item panika;
  \item afektas (trumpalaikis emocinis sprogimas);
  \item stresas (emocinė įtampa);
  \item \gls{frustracija}, gali būti viena iš formų:
    \begin{itemize}
      \item \gls{regresija} (verkimas);
      \item agresija (sukeliančios priežasties sunaikinimas)
      \item tendencija automatizmui;
      \item tolerancija – neišsprendžia problemos, o tik ją atideda.
    \end{itemize}
\end{enumerate}
